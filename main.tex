\documentclass[sbc,authoryear]{sbc2019}

% Define custom environments if not already defined

\setlength{\headheight}{36.90002pt}
\usepackage{fontspec}
\usepackage{graphicx}
\usepackage[misc,geometry]{ifsym}
\usepackage{sectsty}
\usepackage{newtxtext}
\usepackage{fontawesome}
\usepackage{academicons}
\usepackage{color}
\usepackage{aas_macros}
\usepackage{adjustbox}
\usepackage{array}
\usepackage[bottom]{footmisc}
\usepackage{booktabs}
\usepackage{tabularx}
\usepackage{longtable}
\usepackage{supertabular}
\usepackage{afterpage}
\usepackage{url}
\usepackage{enumitem}
\usepackage{pifont}
\usepackage{multicol}
\usepackage{multirow}
\usepackage{IEEEtrantools}
\usepackage{ragged2e}
\usepackage{float}
\setcitestyle{square}
\usepackage{caption}
\usepackage[colorlinks=true, urlcolor=blue]{hyperref}

% --- Float and spacing optimization (template-safe) ---
\setlength{\textfloatsep}{8pt plus 2pt minus 2pt}
\setlength{\floatsep}{8pt plus 2pt minus 2pt}
\setlength{\intextsep}{6pt plus 2pt minus 2pt}
\setlength{\dbltextfloatsep}{8pt plus 2pt minus 2pt}
\setlength{\dblfloatsep}{8pt plus 2pt minus 2pt}
\renewcommand{\topfraction}{0.9}
\renewcommand{\bottomfraction}{0.8}
\renewcommand{\textfraction}{0.1}
\renewcommand{\floatpagefraction}{0.8}
\renewcommand{\dbltopfraction}{0.9}
\renewcommand{\dblfloatpagefraction}{0.8}
\setcounter{topnumber}{4}
\setcounter{bottomnumber}{4}
\setcounter{totalnumber}{8}
\setcounter{dbltopnumber}{4}
\usepackage[normalem]{ulem} % para subrayar



\definecolor{orcidlogo}{rgb}{0.37,0.48,0.13}
\definecolor{unilogo}{rgb}{0.16, 0.26, 0.58}
\definecolor{maillogo}{rgb}{0.58, 0.16, 0.26}
\definecolor{darkblue}{rgb}{0.0,0.0,0.0}
\hypersetup{colorlinks,breaklinks,
            linkcolor=darkblue,urlcolor=darkblue,
            anchorcolor=darkblue,citecolor=darkblue}
%\hypersetup{colorlinks,citecolor=blue,linkcolor=blue,urlcolor=blue}

%%%%%%% IMPORTANT: We disable hyperlinks by default with this line, to avoid the error "\pdfendlink ended up in different nesting level" while writing.
%\hypersetup{draft}

\jid{JBCS}
\jtitle{Journal of Internet Services and Applications, 2023, 13:1, }
\doi{10.5753/jisa.2023.XXXXXX}
\copyrightstatement{This work is licensed under a Creative Commons Attribution 4.0 International License}
\jyear{2023}

\title[Literature Review on Container-Based Virtualization Technologies]{Literature Review on Container-Based Virtualization Technologies}

\author[Arias et al. 2026]{
\affil{\textbf{José A. Arias-Pinzón}~\href{https://orcid.org/0009-0006-1879-1172}{\textcolor{orcidlogo}{\aiOrcid}}~\textcolor{blue}{\faEnvelopeO}~~[~\textbf{Universidad del Quindío}~|\href{mailto:josea.ariasp@uqvirtual.edu.co}{~\textbf{\textit{josea.ariasp@uqvirtual.edu.co}}}~]}

\affil{\textbf{Anobis H. Correa-Urbano}~\href{https://orcid.org/0009-0000-1578-3931}{\textcolor{orcidlogo}{\aiOrcid}}~~[~\textbf{Universidad del Quindío}~|\href{mailto:anobish.correau@uqvirtual.edu.co}{~\textbf{\textit{anobish.correau@uqvirtual.edu.co}}}~]}

\affil{\textbf{Luis E. Sepúlveda-Rodríguez}~\href{https://orcid.org/0000-0003-2446-0602}{\textcolor{orcidlogo}{\aiOrcid}}~~[~\textbf{Universidad del Quindío}~|\href{mailto:lesepulveda@uniquindio.edu.co}{~\textbf{\textit{lesepulveda@uniquindio.edu.co}}}~]}

\affil{\textbf{Christian A. Candela-Uribe}~\href{https://orcid.org/0000-0002-3961-1840}{\textcolor{orcidlogo}{\aiOrcid}}~~[~\textbf{Universidad del Quindío}|\href{mailto:christiancandela@uniquindio.edu.co}{~\textbf{\textit{christiancandela@uniquindio.edu.co}}}~]}

}

\begin{document}

\begin{frontmatter}
\maketitle

\begin{mail}
Facultad de ingeniería, Universidad del Quindío, Carrera 15 \#12N, Armenia, Quindío, 630001, Colombia. 
%Full list of authors' information is available at the end of the article.
\end{mail}

\begin{dates}
\small{\textbf{Received:} DD Month YYYY~~~$\bullet$~~~\textbf{Accepted:} DD Month YYYY~~~$\bullet$~~~\textbf{Published:} DD Month YYYY}
%Full list of authors' information is available at the end of the article.
\end{dates}


\begin{abstract}
\textbf{Abstract}
\noindent Container-based virtualization (CBV) has become a cornerstone of modern IT infrastructure, with technologies such as Docker and Kubernetes dominating application packaging and orchestration, respectively. Despite the proliferation of research in this domain, no existing secondary study systematically maps CBV technologies across both IT domains and academic dimensions—comprising education, research, and outreach. This paper presents a Systematic Mapping Study (SMS) of 226 primary studies published between 2022 and 2024, identified through a hybrid strategy combining database searches across five digital libraries and forward/backward snowballing. Studies were classified using 11 IT domains and three academic dimensions, assessed through three quality indices (CVI, SCI, IRRQ), and organized into a dual-axis taxonomic structure.\\
The results reveal a marked concentration around Docker (41.6\%) and Kubernetes (29.6\%), a dominance of IT Infrastructure as a research domain (75.66\%), and a significant underrepresentation of education (11.06\%) and outreach (5.88\%) in the CBV literature. Six concrete research gaps are identified, including the need for alternative runtime evaluation, orchestration beyond Kubernetes, and empirical studies on CBV in educational contexts. The proposed taxonomy and identified gaps provide a structured foundation for researchers, educators, and practitioners navigating the rapidly expanding CBV landscape.
\end{abstract}

\begin{keywords}
Container-based virtualization, Cloud computing, Docker,  IT infrastructure, Kubernetes, Software engineering, Systematic mapping study.
\end{keywords}

%\begin{license}
%Published under the Creative Commons Attribution 4.0 International Public License (CC BY 4.0)
%\end{license}

\end{frontmatter}

%-----------Section 1---------------
\section{Introduction}\label{sec:intro}
Cloud computing has become a widely adopted paradigm in contemporary information technology, enabling scalable and resilient solutions through on-demand resource provisioning~\citep{falade-segun}. Among the foundational technologies supporting this paradigm, container-based virtualization (CBV) has gained wide adoption due to its lightweight footprint, portability, and rapid deployment capabilities~\citep{mohamed-almoudane}. Unlike full virtualization, which emulates complete hardware environments, containers share the host operating system kernel while maintaining application isolation, resulting in substantially lower overhead~\citep{KOZHIRBAYEV2017175}. These properties have positioned CBV as a commonly used approach for deploying, managing, and scaling applications across distributed environments, facilitating continuous integration, agile development, and microservices architectures~\citep{clemente-mateo}.

Within the CBV ecosystem, Docker has emerged as the \textit{de facto} standard for container creation and management, while Kubernetes dominates container orchestration. However, the rapid evolution of this field has produced a diverse landscape of alternative technologies---including Podman, Singularity, LXC, gVisor, and Kata Containers---each optimized for specific use cases such as rootless operation, high-performance computing, or enhanced security~\citep{BARESI2024111965}. This technological diversification necessitates a systematic analysis to identify which technologies are adopted in which contexts and to what extent.

Beyond industrial applications, CBV technologies have increasingly permeated academic settings, supporting education through reproducible laboratory environments, enabling research through portable computational pipelines, and facilitating outreach through accessible cloud-based platforms. However, the extent and nature of this academic adoption remain poorly characterized in the literature.

This paper presents a Systematic Mapping Study (SMS) that addresses this gap by mapping 226 primary studies (2022--2024) across 11 IT domains and three academic dimensions (education, research, outreach). The study contributes: \textit{(1)}~a comprehensive, reproducible mapping of the CBV research landscape; \textit{(2)}~a proposed dual-axis taxonomic structure linking technologies to both IT domains and academic impact; and \textit{(3)}~the identification of six concrete research gaps with actionable future directions.

The remainder of this paper is organized as follows: Section~\ref{sec:motivation} outlines the study motivation. Section~\ref{sec:related-works} analyzes related works. Section~\ref{sec:method-review} describes the SMS methodology. Section~\ref{sec:threat-validity} addresses threats to validity. Section~\ref{sec:analysis-and-discussion} presents the analysis, discussion, and future research directions. Section~\ref{sec:conclusions} concludes the paper.
%-----------Section 2---------------
\section{Motivation}
\label{sec:motivation}
The adoption of cloud computing has produced a wide range of solutions based on container-based virtualization~\cite{HASSAN2022100138}. However, as noted by multiple authors~\cite{waseem2024containerization,VhatkarBhole2022_Survey_ContainerResourceAllocation,kithulwatta2022integration}, the literature remains fragmented: the high volume of publications makes it difficult to identify clear usage patterns, benefits, and limitations across application domains.

This study is motivated by three specific needs. First, there is no existing secondary study that simultaneously maps CBV technologies across multiple IT domains \textit{and} academic dimensions (education, research, outreach). Second, the rapid proliferation of container technologies beyond Docker---including Podman, Singularity, gVisor, and others---requires a systematic assessment of their adoption and research coverage. Third, the potential of CBV as a transversal tool for academic activities (reproducible research, portable teaching environments, accessible outreach platforms) has not been systematically evaluated.

The expected outcomes include: \textit{(i)}~a comprehensive map of CBV research trends across 11 IT domains; \textit{(ii)}~a classification of studies by their contribution to education, research, and outreach; and \textit{(iii)}~a taxonomic structure that can guide technological decision-making for researchers, educators, and practitioners.
%-----------Section 3---------------
\section{Related Works}
\label{sec:related-works}
Several secondary studies have addressed container-based virtualization (CBV) from different perspectives. To position the contribution of the present SMS, this section critically analyzes the existing literature organized by three dimensions: \textit{(i)}~scope and coverage, \textit{(ii)}~methodology, and \textit{(iii)}~domain focus. Table~\ref{tab:related-works-comparison} synthesizes this comparative analysis.

Prior reviews differ substantially in the breadth of technologies and application contexts examined. \cite{Bentaleb2022} and~\cite{SepulvedaRodriguez2022} both propose taxonomic classifications of virtualization technologies; however, neither extends its analysis to the academic dimension (education, research, and outreach). Similarly,~\cite{Malhotra2024} provide a systematic literature review focused exclusively on container lifecycle management---image detection, scheduling, security, and performance---without mapping these concerns to specific IT domains or educational contexts. In contrast,~\cite{Kaiser2022,Kaiser2023} narrow their scope to ARM-compatible container technologies, prioritizing energy efficiency and edge computing performance. While valuable, this architectural focus limits the generalizability of their findings across the full spectrum of CBV use cases.

Among the related works, only~\cite{10094059} adopt a systematic mapping methodology, focusing on container orchestration architectures in cloud computing. Their categorization scheme, however, is restricted to orchestration and does not encompass runtime technologies, academic applications, or cross-domain analysis. The remaining studies employ narrative or traditional review methods, which---while informative---lack the structured, reproducible search protocols and quality assessment mechanisms that characterize an SMS~\cite{Kitchenham2010}.

A common limitation across all reviewed studies is the absence of a cross-cutting analysis that maps CBV technologies to both IT domains (e.g., software development, HPC, security, AI) and academic dimensions simultaneously. None of the existing works:~\textit{(a)}~provides a comprehensive mapping of CBV across multiple IT domains;~\textit{(b)}~examines the role of containerization in education and outreach;~or~\textit{(c)}~offers a taxonomic structure linking technologies, domains, and academic impact.

\begin{table}[htbp]
\centering
\scriptsize
\renewcommand{\arraystretch}{1.1}
\begin{tabularx}{\columnwidth}{>{\raggedright\arraybackslash}p{1.5cm} >{\centering\arraybackslash}p{0.7cm} >{\centering\arraybackslash}p{0.9cm} >{\centering\arraybackslash}p{0.7cm} >{\centering\arraybackslash}p{0.9cm} >{\centering\arraybackslash}p{0.9cm}}
\toprule
\textbf{Study} & \textbf{Type} & \textbf{Multi-domain} & \textbf{Acad.} & \textbf{Taxon.} & \textbf{Reprod.} \\
\midrule
\cite{Bentaleb2022} & Review & \ding{55} & \ding{55} & \ding{51} & \ding{55} \\
\cite{Kaiser2022} & Review & \ding{55} & \ding{55} & \ding{55} & \ding{55} \\
\cite{SepulvedaRodriguez2022} & Review & \ding{55} & \ding{55} & \ding{51} & \ding{55} \\
\cite{Kaiser2023} & Review & \ding{55} & \ding{55} & \ding{55} & \ding{55} \\
\cite{10094059} & SMS & \ding{55} & \ding{55} & \ding{51} & Partial \\
\cite{Malhotra2024} & SLR & \ding{55} & \ding{55} & \ding{55} & Partial \\
\textbf{This study} & \textbf{SMS} & \ding{51} & \ding{51} & \ding{51} & \ding{51} \\
\bottomrule
\end{tabularx}
\caption{Comparative analysis of related secondary studies. Multi-domain: covers multiple IT domains; Acad.: includes academic dimensions; Taxon.: proposes a taxonomy; Reprod.: provides reproducibility artifacts.}\label{tab:related-works-comparison}
\end{table}

This SMS addresses these gaps by providing:~\textit{(1)}~a systematic, reproducible mapping of 226 primary studies across 11 IT domains;~\textit{(2)}~a novel classification linking CBV technologies to education, research, and outreach; and~\textit{(3)}~a taxonomic structure that integrates technological and academic perspectives, enabling researchers and practitioners to identify both consolidated areas and under-explored research opportunities.
%-------------Section 4--------------------------
%-----------Section 4---------------
\section{Review Method}
\label{sec:method-review}
This study follows a Systematic Mapping Study (SMS) methodology, guided by the established frameworks of Petersen et al.~\cite{Runeson2009} and Kitchenham and Charters~\cite{Kitchenham2010}. Unlike a Systematic Literature Review (SLR), which aims to synthesize evidence on a specific question, an SMS provides a broad overview of a research area by classifying and categorizing existing literature~\cite{Runeson2009}. Following \cite{Mourao2017} and \cite{Nguyen2015}, a hybrid search approach was adopted, combining automated database queries with manual snowballing to maximize coverage.

To ensure transparency and reproducibility, the SMS-Builder tool~\cite{Candela-Uribe2022} was employed throughout the process for study identification, classification, data extraction, and quality assessment. The SMS process comprises six stages: (1)~planning, (2)~study search, (3)~quality assessment, (4)~data extraction, (5)~study classification, and (6)~results. Figure~\ref{fig:stages-SMS} illustrates these stages.

\begin{figure}[htbp]
    \centering
    \includegraphics[width=0.5\textwidth]{resources/images/planeacion/planeacion.png}
    \caption{Stages of the SMS process}\label{fig:stages-SMS}
\end{figure}

%-----------Section 4.1---------------
\subsection{Planning}\label{subsec:planeacion}
The planning stage established the research goals, questions, metrics, classification criteria, and quality assessment indices. Figure~\ref{fig:etapa1} summarizes the components of this stage.

\begin{figure}[htbp]
    \centering
    \includegraphics[width=0.5\textwidth]{resources/images/planeacion/etapa1.png}
    \caption{Composition of the planning stage}\label{fig:etapa1}
\end{figure}

%-----------Section 4.1.1---------------
\subsubsection{Study Goals}
Two overarching goals were defined to guide this SMS, as presented in Table~\ref{tab:metas-del-estudio}.

\begin{table}[htbp]
    \centering
    \begin{tabular}{>{\centering\arraybackslash}m{1cm} >{\arraybackslash}m{7cm}}
        \hline
        \textbf{Goal} & \textbf{Description} \\
        \hline\\
        G1 & Identify studies related to CBV in education, research, and outreach. \\
        \\
        G2 & Classify studies related to CBV across IT domains, including software development, computational thinking, parallel computing, data analysis, artificial intelligence, computer networks, IT infrastructure, HPC, security, cloud computing, and blockchain. \\\\
        
        \hline
    \end{tabular}
    \caption{Goals of the study}\label{tab:metas-del-estudio}
\end{table}

\begin{table*}[!t]
\centering

\renewcommand{\arraystretch}{1.4}
\begin{tabularx}{\textwidth}{>{\centering\arraybackslash}m{0.05\textwidth} >{\centering\arraybackslash}m{0.08\textwidth} >{\RaggedRight\arraybackslash}X >{\RaggedRight\arraybackslash}X}
\toprule
\textbf{Goal} & \textbf{Question} & \textbf{Description} & \textbf{Motivation} \\
\midrule
G1 & Q1 & Which studies related to container-based virtualization (CBV) technologies contribute to education, research, and outreach? & CBV enables environment reproducibility, facilitating the transfer of IT solutions across contexts. Understanding its academic penetration can stimulate cross-domain innovation. \\
\midrule
G2 & Q2 & Which primary studies related to CBV technologies contribute to IT domains such as software development, HPC, AI, security, cloud computing, and others? & The goal is to provide a structured overview of CBV adoption across IT domains, enabling researchers and practitioners to identify trends without requiring exhaustive primary analysis. \\
\bottomrule
\end{tabularx}
\caption{Research questions and their motivation}\label{tab:descripcion-preguntas}
\end{table*}
%------Section 4.1.2-----------------------
\subsubsection{Research Questions}
\label{sec:research-question}
The research questions were formulated using the GQM (\textit{Goal Question Metric}) framework~\cite{Needleman2002} and the PICOC model~\cite{petticrew2008systematic}, which structures the population, intervention, comparison, outcome, and context of the study (Table~\ref{tab:aspectos-PICOC}). Two research questions were derived, as detailed in Table~\ref{tab:descripcion-preguntas}.

%------Section 4.1.3-----------------------
\subsubsection{Metrics}
\label{sec:metrics}
Quantitative metrics were defined to measure the distribution of studies across the classification structure (Table~\ref{tab:metricas}). The search period was restricted to 2022--2024 to ensure currency.

\begin{table}[H]
    \scriptsize % reduce text size
    \centering
    \renewcommand{\arraystretch}{1.3}
    \begin{tabularx}{\columnwidth}{>{\centering\arraybackslash}m{0.18\columnwidth} >{\RaggedRight\arraybackslash}X}
        \hline
        \textbf{Aspect} & \textbf{Description} \\
        \hline
        Population & Studies related to CBV applied across IT domains, with emphasis on education, research, and outreach. \\
        Intervention & Identification and classification of CBV studies within established IT domains. \\
        Comparison & 
        1. Comparison of CBV projects by reported success rates across IT domains. \newline
        2. Analysis of CBV impact on academic activities relative to alternative solutions. \\
        Outcome & Classification structure mapping CBV studies to IT domains and academic dimensions. \\
        Context & Education, research, and outreach contexts adopting CBV technologies. \\
        \hline
    \end{tabularx}
    \caption{PICOC model specification}\label{tab:aspectos-PICOC}
\end{table}

%------Section 4.1.4-----------------------
\subsubsection{Research Topics}
\label{sec:research-topics}
Based on the research questions and PICOC model, four research topics were defined: \textit{Container-based virtualization}, \textit{Education}, \textit{Research}, and \textit{Industry}. These topics were further refined through the IT domains identified as relevant to the study scope.

\subsubsection{Inclusion and Exclusion Criteria}
\begin{table*}[!t]
\centering
\renewcommand{\arraystretch}{1.4}
\begin{tabularx}{\textwidth}{>{\centering\arraybackslash}m{0.15\textwidth} >{\RaggedRight\arraybackslash}X >{\RaggedRight\arraybackslash}X}
\toprule
\textbf{Category} & \textbf{Inclusion} & \textbf{Exclusion} \\
\midrule
Screening field & Abstract & --- \\
\midrule
Publication type & Journal articles and conference proceedings & Theses, book chapters, grey literature \\
\midrule
Discipline & Computer Science, Information Technology, Engineering, IT Management & Disciplines unrelated to virtualization or computing \\
\midrule
Time period & 2022--2024 & Before 2022 \\
\midrule
Language & English & Non-English publications \\
\bottomrule
\end{tabularx}
\caption{Inclusion and exclusion criteria}\label{tab:criterios}
\end{table*}

\label{sec:inclusion-exclusion}
Table~\ref{tab:criterios} presents the inclusion and exclusion criteria. The three-year window (2022--2024) balances currency with sufficient volume. Both journal articles and conference proceedings were included to capture the full publication landscape in this rapidly evolving field.\\

\begin{table}[H]
\centering
\renewcommand{\arraystretch}{1.3}
\begin{tabularx}{\columnwidth}{>{\centering\arraybackslash}m{0.15\textwidth} >{\RaggedRight\arraybackslash}X}
\toprule
\textbf{Metric} & \textbf{Description} \\
\midrule
M1 & Number of studies identified per IT domain. \\
M2 & Number of studies classified under education. \\
M3 & Number of studies classified under research. \\
M4 & Number of studies classified under outreach. \\
\bottomrule
\end{tabularx}
\caption{Metrics defined for the analysis}\label{tab:metricas}
\end{table}

%------Section 4.1.6-----------------------
\subsubsection{Quality Assessment Criteria}
Three quality indices were defined to evaluate the relevance and rigor of the selected studies.

\paragraph{Content Validity Index (CVI).}
The CVI assesses the degree to which each study aligns with the SMS objectives, adapted from established content validity methodology~\cite{almanasreh2019evaluation,yaghmaei2003content}. Each study was independently rated by $K$~evaluators (where $K$ is odd, to prevent ties) on a scale from 0~(no relevance) to~5 (high relevance). Following the proportion-based CVI approach~\cite{almanasreh2019evaluation}, we define the item-level CVI (I-CVI) as the proportion of evaluators who rate a study above a relevance threshold $t$:
\begin{equation}
\label{eq:cvi}
\text{I-CVI} = \frac{n_{t}}{K}
\end{equation}
where $n_{t}$ is the number of evaluators assigning a score $\geq t$ (with $t = 3$ adopted in this study), and $K$ is the total number of evaluators. An I-CVI $\geq 0.78$ indicates acceptable content validity~\cite{almanasreh2019evaluation}. For aggregation across the study corpus, the Scale-level CVI based on the average method (S-CVI/Ave) is computed as the mean of all I-CVI values.

\paragraph{Scientific Citation Index (SCI).}
The SCI captures citation-normalized impact relative to publication recency. For a study with $C$~citations accumulated between 2022 and 2024, published $A$~years before the extraction date:
\begin{equation}
\label{eq:sci}
\text{SCI} = \frac{C}{A}
\end{equation}
This normalization ensures that recently published studies with emerging citation counts are not penalized relative to older, more-cited works.

\paragraph{Index of Relationship to Research Questions (IRRQ).}
The IRRQ quantifies the coverage of a study with respect to the defined research questions. Given $Q = 2$ research questions in this SMS, the IRRQ for a study addressing $n$ questions is:
\begin{equation}
\label{eq:irrq}
\text{IRRQ} = \frac{n}{Q}
\end{equation}
where $n \in \{0, 1, 2\}$ and $Q = 2$. Thus, IRRQ $\in \{0, 0.5, 1\}$, where 1~indicates full coverage of both research questions. This index enables identification of studies with broad versus narrow thematic relevance.

% subsection 4.2
\subsection{Stage 2: Study Search}
A hybrid search strategy combining database queries and snowballing was employed. Figure~\ref{fig:etapa2} summarizes the components of this stage.
\begin{figure}[htbp]
    \centering
    \includegraphics[width=0.5\textwidth]{resources/images/planeacion/estrategias-busqueda.png}
    \caption{Composition of the study search stage}
    \label{fig:etapa2}
\end{figure}

%sub-subsection 4.2.1 
\subsubsection{Defining the Search Strategy}\label{subsubsec:Definiendo la Estrategia de Busqueda}
Two complementary strategies were combined. The first involves automated search string execution in academic databases~\cite{jalali2012systematic}. The second, snowballing, identifies additional studies through backward (reference tracking) and forward (citation tracking) analysis of a seed set~\cite{jalali2012systematic,goodman1961snowball}.

%sub-subsection 4.2.2
\subsubsection{Search Strategy 1: Databases}
This strategy comprises two phases: \textit{Study Identification} (search string construction and execution) and \textit{Study Selection} (criteria-based refinement).

\begin{itemize}
    \item \textbf{Study Identification:} Five databases were queried: \textit{ACM}, \textit{IEEE Xplore}, \textit{Springer}, \textit{Science Direct}, and \textit{Taylor \& Francis}. Keywords were derived from the PICOC model (Table~\ref{tab:palabras-clave}) and expanded with synonyms (Table~\ref{tab:keywords}). Boolean operators (\textit{AND}, \textit{OR}) and exact-phrase matching were used to construct database-specific search strings through iterative pilot searches. The complete search strings are available via the reproducibility artifacts (Section~\ref{sec:reproducibility-review-method}).
    
    Execution across all databases yielded \textbf{6,530} preliminary results, with Springer contributing the largest share (\textbf{4,562}; 69.8\%). Table~\ref{tab:resultados-busqueda-sin-criterio} details the distribution.

    \begin{table}[H]
        \caption{Keywords identified using the PICOC model}
        \label{tab:palabras-clave}
        \scriptsize
        \centering
        \setlength{\tabcolsep}{4pt}
        \renewcommand{\arraystretch}{1.05}
        \begin{tabularx}{\columnwidth}{@{}l>{\RaggedRight\arraybackslash}X@{}}
            \toprule
            \textbf{Aspect} & \textbf{Description} \\
            \midrule
            Population & CBV, IT Domains, Education, Research, Outreach \\
            Intervention & Identification, Classification \\
            Comparison & Success rate, Evidence of use \\
            Output & Classification of CBV studies per IT domain \\
            Context & Education, Research, Outreach \\
            \bottomrule
        \end{tabularx}
    \end{table}

\begin{table}[H]
    \caption{Keywords for database search}
    \label{tab:keywords}
    \scriptsize
    \centering
    \setlength{\tabcolsep}{4pt}
    \renewcommand{\arraystretch}{1.05}
    \begin{tabularx}{\columnwidth}{@{}l>{\RaggedRight\arraybackslash}X@{}}
        \toprule
        \textbf{Keyword} & \textbf{Synonyms} \\
        \midrule
        Container-based virtualization & Application virtualization, Docker, Lightweight Virtualization \\
        Education & Education System, Education Development, Higher Education \\
        Research & Research Group, Research Proposal \\
        Industry & IT Services, Technology Infrastructure, Cloud Computing \\
        \bottomrule
    \end{tabularx}
\end{table}

    \item \textbf{Study Selection:} Application of inclusion and exclusion criteria reduced the set to \textbf{976} studies (Table~\ref{tab:resultados-busqueda-criterios}), with Springer maintaining the largest contribution (\textbf{592}; 60.65\%). After removing \textbf{274} duplicates, a screening process (title, abstract, and keyword review) excluded \textbf{593} irrelevant studies, yielding \textbf{110} selected studies from the database strategy. Figure~\ref{fig:resumen-busqueda-bds} summarizes this process.
\end{itemize}
\begin{figure}[htbp]
    \centering
    \includegraphics[width=0.5\textwidth]{resources/images/busqueda-estudios/busqueda-bd.png}
    \caption{Summary of activities and results obtained in the database search strategy}
    \label{fig:resumen-busqueda-bds}
\end{figure}

\begin{table*}[tbp]
    \caption{Search results per database using keywords}\label{tab:resultados-busqueda-sin-criterio}
    \scriptsize
    \centering
    \setlength{\tabcolsep}{4pt}
    \renewcommand{\arraystretch}{1.05}
    \begin{tabular}{@{}lrrrrr@{}}
        \toprule
        \textbf{Criterion} & \textbf{ACM} & \textbf{IEEE} & \textbf{Science Direct} & \textbf{Springer} & \textbf{Taylor and Francis} \\
        \midrule
        Search results using keywords only & 189 & 426 & 4562 & 353 & 1000 \\
        Contribution percentage & 2.89\% & 6.52\% & 69.86\% & 5.4\% & 15.31\% \\
        \bottomrule
    \end{tabular}
\end{table*}

\begin{table*}[tbp]
    \caption{Search results per database using keywords after applying inclusion/exclusion criteria}\label{tab:resultados-busqueda-criterios}
    \scriptsize
    \centering
    \setlength{\tabcolsep}{4pt}
    \renewcommand{\arraystretch}{1.05}
    \begin{tabular}{@{}lrrrrr@{}}
        \toprule
        \textbf{Criterion} & \textbf{ACM} & \textbf{IEEE} & \textbf{Science Direct} & \textbf{Springer} & \textbf{Taylor and Francis} \\
        \midrule
        Search results after applying keywords only & 48 & 134 & 46 & 592 & 156 \\
        Contribution percentage & 4.91\% & 13.72\% & 4.71\% & 60.65\% & 15.98\% \\
        \bottomrule
    \end{tabular}
\end{table*}

% sub-subsection 4.2.3
\subsubsection{Search Strategy 2: Snowballing}
The snowballing search strategy began with the identification of the base set of articles. This base set was obtained from Search Strategy 1. The procedure consisted of two phases:

The first phase, called \textit{Baseline Construction}, aimed to establish the articles on which a citation and reference analysis would be performed. To form this initial set of studies, several criteria were applied, including the CVI \textit{(Content Validity Index)}, the SCI \textit{(Scientific Citation Index)}, and the direct inclusion criterion. The second phase, called \textit{Study Selection}, focused on the analysis of references \textit{(Backward Snowballing)} and citations \textit{(Forward Snowballing)} corresponding to each article \cite{10.1145/2601248.2601268}.

Baseline construction started from the \textbf{110} articles obtained through the database search strategy. From this set, \textbf{25} articles were selected using the SCI quality criterion. This criterion was chosen because it relies on the number of citations received by each article rather than on evaluator judgment, thus providing an objective indicator of academic impact. Selection was performed through citation frequency analysis, extracting the first quartile \textbf{(Q1)} corresponding to the most-cited articles.
As part of the SMS process, studies can also be incorporated through direct inclusion, whereby an article previously known to the authors is added without originating from a database search. This procedure adds flexibility by allowing the integration of works deemed relevant to the research objectives. In this case, one article was incorporated through direct inclusion, bringing the total to \textbf{26} articles in the baseline.

After baseline construction, reference analysis was performed. The forward search was conducted using Google Scholar following the practices described in~\cite{8747000}, identifying a total of \textbf{495} new articles. The backward search yielded \textbf{87} additional articles.

\textbf{14} duplicate articles were removed from the backward and forward search results. Subsequently, the \textit{Screening} process was applied again, consisting of title, abstract, and keyword review as in the previous phase. This procedure reduced the set to \textbf{116} articles selected through the snowballing search strategy. Figure~\ref{fig:resumen-busqueda-snowballing} presents a summary of the process followed in this search strategy.

% sub-subsection 4.2.4
\subsubsection{Results of the Study Search}\label{subsubsec:resultados-busqueda}
The combined search yielded \textbf{226} primary studies: \textbf{110} from databases, \textbf{115} from snowballing, and \textbf{1} through direct inclusion. The nearly-equal split between strategies (Table~\ref{tab:resultados-busqueda}) confirms the complementarity of the hybrid approach.
\begin{table}[H]
\caption{Results of the study search}\label{tab:resultados-busqueda}
\centering
\scriptsize
\setlength{\tabcolsep}{4pt}
\renewcommand{\arraystretch}{1.05}
\begin{tabular}{@{}lrr@{}}
\toprule
\textbf{Strategy} & \textbf{Studies} & \textbf{\%} \\
\midrule
Databases & 110 & 48.67\% \\
Snowballing & 115 & 50.88\% \\
Direct Inclusion & 1 & 0.44\% \\
\textbf{Total} & \textbf{226} & \textbf{100\%} \\
\bottomrule
\end{tabular}
\end{table}
\begin{figure}[htbp]
    \centering
    \includegraphics[width=0.5\textwidth]{resources/images/busqueda-estudios/busqueda-snowball.png}
    \caption{Summary of the snowballing search strategy}
    \label{fig:resumen-busqueda-snowballing}
\end{figure}

% subseccion 4.3
\subsection{Stage 3: Quality Assessment}
Although quality assessment is not mandatory in an SMS~\cite{8747000}, incorporating it strengthens the rigor of the mapping and brings it closer to a systematic review~\cite{10.1145/2601248.2601268}. Three complementary indices---CVI, SCI, and IRRQ---were applied to evaluate study relevance.

%sub-subseccion 4.3.1
\subsubsection{Content Validity Assessment (CVI)}
Each study was independently rated by an odd number of evaluators ($K \geq 3$) on a 0--5 relevance scale. The proportion-based I-CVI (Equation~\ref{eq:cvi}) was computed for each study, and studies with I-CVI $\geq 0.78$ were considered to have acceptable content validity. Two assessment rounds were conducted: the first during baseline construction for snowballing (Section~\ref{subsubsec:resultados-busqueda}), and the second after all 226 studies were identified, with results reported in Section~\hyperref[sec:clasificacion-estudios]{Study Classification}.

%sub-subseccion 4.3.2
\subsubsection{Citation-Based Quality Assessment (SCI)}
The SCI (Equation~\ref{eq:sci}) was computed using citation data from Google Scholar and the SMS-Builder tool~\cite{candela2020smsbuilder}. A frequency analysis identified the top quartile (Q1) of studies by SCI, representing those with the highest citation-normalized impact.

%sub-subseccion 4.3.3
\subsubsection{Research Question Coverage Assessment (IRRQ)}
The IRRQ (Equation~\ref{eq:irrq}) was computed for each study based on its thematic alignment with Q1 and Q2, as determined through the classification process. Studies with IRRQ~$= 1$ (addressing both research questions) were identified through frequency analysis as the most thematically comprehensive.

% subseccion 4.4
\subsection{Stage 4: Data Extraction}
After completing study search and quality assessment, \textbf{226} primary studies were identified and labeled SPS001 through SPS226. The complete list is provided in Table~\ref{tab:sps-list}.

\subsection{Stage 5: Study Classification}\label{sec:clasificacion-estudios}
The SPS were classified using the topics defined during planning (Table~\ref{tab:clasificacion}). A single SPS may be associated with multiple topics; for example, SPS069 is classified under IT Infrastructure, Security, Cloud Computing, and Research. This multi-label approach reflects the interdisciplinary nature of CBV research and enables cross-domain analysis.

Following classification, each SPS was evaluated using the CVI, SCI, and IRRQ quality indices. Tables~\ref{tab:higher-cvi}, \ref{tab:higher-sci}, and \ref{tab:higher-irrq} present the top-quartile studies for each index, disaggregated by topic and year.

% subseccion 4.6
\subsection{Stage 6: Results}
This section presents and interprets the findings from the SMS, organized in three parts: (1)~an overview of the SPS corpus with source and temporal analysis, (2)~a technology and domain distribution analysis with interpretation of observed trends, and (3)~a keyword co-occurrence analysis.

%sub-subseccion 4.6.1
\subsubsection{Overview of the SPS Corpus}
The SMS identified \textbf{226} selected primary studies (SPS), listed in Table~\ref{tab:sps-list}. The following subsections analyze the distribution of these studies across sources, technologies, academic dimensions, and quality indices.

\paragraph{Source and strategy distribution.}
Table~\ref{tab:clasificacion-topicos} presents the classification of CBV technologies by academic dimension. Docker dominates across education, research, and outreach, accounting for \textbf{100} SPS (44.24\%). This predominance reflects Docker's mature ecosystem, extensive documentation, and low barrier to entry, which collectively facilitate adoption in academic contexts where ease of deployment is prioritized over specialized performance characteristics.\\
Table~\ref{tab:clasificacion-sps} maps IT domains to academic dimensions. IT Infrastructure is the most represented domain, with \textbf{171} SPS (75.66\%), indicating that containerization research remains strongly anchored to infrastructure-level concerns such as deployment, scaling, and resource management.
\onecolumn
\begin{longtable*}{c>{\centering\arraybackslash}p{6cm}
                 c>{\centering\arraybackslash}p{6cm}}
\label{tab:sps-list} \\
\textbf{ID} & \textbf{Ref} &
\textbf{ID} & \textbf{Ref} \\
\hline
\endfirsthead

\multicolumn{4}{c}{{\bfseries \tablename\ \thetable{} -- continued}} \\
\textbf{ID} & \textbf{Ref} &
\textbf{ID} & \textbf{Ref} \\
\hline
\endhead

\hline \multicolumn{4}{r}{{Continues on the next page}} \\
\endfoot

\hline
\endlastfoot
SPS001 & \citeauthor{1} & SPS002 & \citeauthor{2} \\
SPS003 & \citeauthor{3} & SPS004 & \citeauthor{4} \\
SPS005 & \citeauthor{5} & SPS006 & \citeauthor{6} \\
SPS007 & \citeauthor{7} & SPS008 & \citeauthor{8} \\
SPS009 & \citeauthor{9} & SPS010 & \citeauthor{10} \\
SPS011 & \citeauthor{11} & SPS012 & \citeauthor{12} \\
SPS013 & \citeauthor{13} & SPS014 & \citeauthor{14} \\
SPS015 & \citeauthor{15} & SPS016 & \citeauthor{16} \\
SPS017 & \citeauthor{17} & SPS018 & \citeauthor{18} \\
SPS019 & \citeauthor{19} & SPS020 & \citeauthor{20} \\
SPS021 & \citeauthor{21} & SPS022 & \citeauthor{22} \\
SPS023 & \citeauthor{23} & SPS024 & \citeauthor{24} \\
SPS025 & \citeauthor{25} & SPS026 & \citeauthor{26} \\
SPS027 & \citeauthor{27} & SPS028 & \citeauthor{28} \\
SPS029 & \citeauthor{29} & SPS030 & \citeauthor{30} \\
SPS031 & \citeauthor{31} & SPS032 & \citeauthor{32} \\
SPS033 & \citeauthor{33} & SPS034 & \citeauthor{34} \\
SPS035 & \citeauthor{35} & SPS036 & \citeauthor{36} \\
SPS037 & \citeauthor{37} & SPS038 & \citeauthor{38} \\
SPS039 & \citeauthor{39} & SPS040 & \citeauthor{40} \\
SPS041 & \citeauthor{41} & SPS042 & \citeauthor{42} \\
SPS043 & \citeauthor{43} & SPS044 & \citeauthor{44} \\
SPS045 & \citeauthor{45} & SPS046 & \citeauthor{46} \\
SPS047 & \citeauthor{47} & SPS048 & \citeauthor{48} \\
SPS049 & \citeauthor{49} & SPS050 & \citeauthor{50} \\
SPS051 & \citeauthor{51} & SPS052 & \citeauthor{52} \\
SPS053 & \citeauthor{53} & SPS054 & \citeauthor{54} \\
SPS055 & \citeauthor{55} & SPS056 & \citeauthor{56} \\
SPS057 & \citeauthor{57} & SPS058 & \citeauthor{58} \\
SPS059 & \citeauthor{59} & SPS060 & \citeauthor{60} \\
SPS061 & \citeauthor{61} & SPS062 & \citeauthor{62} \\
SPS063 & \citeauthor{63} & SPS064 & \citeauthor{64} \\
SPS065 & \citeauthor{65} & SPS066 & \citeauthor{66} \\
SPS067 & \citeauthor{67} & SPS068 & \citeauthor{68} \\
SPS069 & \citeauthor{69} & SPS070 & \citeauthor{70} \\
SPS071 & \citeauthor{71} & SPS072 & \citeauthor{72} \\
SPS073 & \citeauthor{73} & SPS074 & \citeauthor{74} \\
SPS075 & \citeauthor{75} & SPS076 & \citeauthor{76} \\
SPS077 & \citeauthor{77} & SPS078 & \citeauthor{78} \\
SPS079 & \citeauthor{79} & SPS080 & \citeauthor{80} \\
SPS081 & \citeauthor{81} & SPS082 & \citeauthor{82} \\
SPS083 & \citeauthor{83} & SPS084 & \citeauthor{84} \\
SPS085 & \citeauthor{85} & SPS086 & \citeauthor{86} \\
SPS087 & \citeauthor{87} & SPS088 & \citeauthor{88} \\
SPS089 & \citeauthor{89} & SPS090 & \citeauthor{90} \\
SPS091 & \citeauthor{91} & SPS092 & \citeauthor{92} \\
SPS093 & \citeauthor{93} & SPS094 & \citeauthor{94} \\
SPS095 & \citeauthor{95} & SPS096 & \citeauthor{96} \\
SPS097 & \citeauthor{97} & SPS098 & \citeauthor{98} \\
SPS099 & \citeauthor{99} & SPS100 & \citeauthor{100} \\
SPS101 & \citeauthor{101} & SPS102 & \citeauthor{102} \\
SPS103 & \citeauthor{103} & SPS104 & \citeauthor{104} \\
SPS105 & \citeauthor{105} & SPS106 & \citeauthor{106} \\
SPS107 & \citeauthor{107} & SPS108 & \citeauthor{108} \\
SPS109 & \citeauthor{109} & SPS110 & \citeauthor{110} \\
SPS111 & \citeauthor{111} & SPS112 & \citeauthor{112} \\
SPS113 & \citeauthor{113} & SPS114 & \citeauthor{114} \\
SPS115 & \citeauthor{115} & SPS116 & \citeauthor{116} \\
SPS117 & \citeauthor{117} & SPS118 & \citeauthor{118} \\
SPS119 & \citeauthor{119} & SPS120 & \citeauthor{120} \\
SPS121 & \citeauthor{121} & SPS122 & \citeauthor{122} \\
SPS123 & \citeauthor{123} & SPS124 & \citeauthor{124} \\
SPS125 & \citeauthor{125} & SPS126 & \citeauthor{126} \\
SPS127 & \citeauthor{127} & SPS128 & \citeauthor{128} \\
SPS129 & \citeauthor{129} & SPS130 & \citeauthor{130} \\
SPS131 & \citeauthor{131} & SPS132 & \citeauthor{132} \\
SPS133 & \citeauthor{133} & SPS134 & \citeauthor{134} \\
SPS135 & \citeauthor{135} & SPS136 & \citeauthor{136} \\
SPS137 & \citeauthor{137} & SPS138 & \citeauthor{138} \\
SPS139 & \citeauthor{139} & SPS140 & \citeauthor{140} \\
SPS141 & \citeauthor{141} & SPS142 & \citeauthor{142} \\
SPS143 & \citeauthor{143} & SPS144 & \citeauthor{144} \\
SPS145 & \citeauthor{145} & SPS146 & \citeauthor{146} \\
SPS147 & \citeauthor{147} & SPS148 & \citeauthor{148} \\
SPS149 & \citeauthor{149} & SPS150 & \citeauthor{150} \\
SPS151 & \citeauthor{151} & SPS152 & \citeauthor{152} \\
SPS153 & \citeauthor{153} & SPS154 & \citeauthor{154} \\
SPS155 & \citeauthor{155} & SPS156 & \citeauthor{156} \\
SPS157 & \citeauthor{157} & SPS158 & \citeauthor{158} \\
SPS159 & \citeauthor{159} & SPS160 & \citeauthor{160} \\
SPS161 & \citeauthor{161} & SPS162 & \citeauthor{162} \\
SPS163 & \citeauthor{163} & SPS164 & \citeauthor{164} \\
SPS165 & \citeauthor{165} & SPS166 & \citeauthor{166} \\
SPS167 & \citeauthor{167} & SPS168 & \citeauthor{168} \\
SPS169 & \citeauthor{169} & SPS170 & \citeauthor{170} \\
SPS171 & \citeauthor{171} & SPS172 & \citeauthor{172} \\
SPS173 & \citeauthor{173} & SPS174 & \citeauthor{174} \\
SPS175 & \citeauthor{175} & SPS176 & \citeauthor{176} \\
SPS177 & \citeauthor{177} & SPS178 & \citeauthor{178} \\
SPS179 & \citeauthor{179} & SPS180 & \citeauthor{180} \\
SPS181 & \citeauthor{181} & SPS182 & \citeauthor{182} \\
SPS183 & \citeauthor{183} & SPS184 & \citeauthor{184} \\
SPS185 & \citeauthor{185} & SPS186 & \citeauthor{186} \\
SPS187 & \citeauthor{187} & SPS188 & \citeauthor{188} \\
SPS189 & \citeauthor{189} & SPS190 & \citeauthor{190} \\
SPS191 & \citeauthor{191} & SPS192 & \citeauthor{192} \\
SPS193 & \citeauthor{193} & SPS194 & \citeauthor{194} \\
SPS195 & \citeauthor{195} & SPS196 & \citeauthor{196} \\
SPS197 & \citeauthor{197} & SPS198 & \citeauthor{198} \\
SPS199 & \citeauthor{199} & SPS200 & \citeauthor{200} \\
SPS201 & \citeauthor{201} & SPS202 & \citeauthor{202} \\
SPS203 & \citeauthor{203} & SPS204 & \citeauthor{204} \\
SPS205 & \citeauthor{205} & SPS206 & \citeauthor{206} \\
SPS207 & \citeauthor{207} & SPS208 & \citeauthor{208} \\
SPS209 & \citeauthor{209} & SPS210 & \citeauthor{210} \\
SPS211 & \citeauthor{211} & SPS212 & \citeauthor{212} \\
SPS213 & \citeauthor{213} & SPS214 & \citeauthor{214} \\
SPS215 & \citeauthor{215} & SPS216 & \citeauthor{216} \\
SPS217 & \citeauthor{217} & SPS218 & \citeauthor{218} \\
SPS219 & \citeauthor{219} & SPS220 & \citeauthor{220} \\
SPS221 & \citeauthor{221} & SPS222 & \citeauthor{222} \\
SPS223 & \citeauthor{223} & SPS224 & \citeauthor{224} \\
SPS225 & \citeauthor{225} & SPS226 & \citeauthor{226} \\
\end{longtable*}
\twocolumn
\paragraph{Technology landscape analysis.}
Figure~\ref{fig:SPS-SOURCE-SE} shows the distribution of studies by source and strategy. Of the database-sourced studies (110~SPS), IEEE Xplore and ACM Digital Library jointly contribute 68.18\%, reflecting the strong alignment of CBV research with computing-focused venues. The snowballing strategy (115~SPS) was dominated by forward snowballing (92.17\%), suggesting that CBV is an expanding field where newer publications actively cite foundational works.
\begin{figure}[htpb]
    \centering
    \includegraphics[width=0.5\textwidth]{resources/images/resultados/SPS-SOURCE-SE.png}
    \caption{SPS by source and search strategy}\label{fig:SPS-SOURCE-SE}
\end{figure}

Figure~\ref{fig:SPS-VBC} reveals the distribution of container runtime technologies. Docker leads with \textbf{94} SPS, followed distantly by Podman (7), LXC and Containerd (4 each), and Singularity, runC, and gVisor (3 each). This concentration raises an important finding: \textit{despite the growing ecosystem of alternative container runtimes designed for security (gVisor, Kata Containers), HPC (Singularity), and rootless operation (Podman), the research community remains heavily Docker-centric}. This gap between technological diversity and research coverage represents an opportunity for future studies to evaluate emerging runtimes in domains where Docker's limitations are well documented.

\begin{center}
    \includegraphics[width=0.5\textwidth]{resources/images/resultados/SPS-VBC.png}
    \captionof{figure}{Distribution of container runtime technologies across SPS}
    \label{fig:SPS-VBC}
\end{center}
Figure~\ref{fig:SPS-ORCH} shows orchestrator distribution. Kubernetes dominates with \textbf{67} SPS, confirming its status as the \textit{de facto} standard for container orchestration. Docker Swarm (9~SPS) and Apache Mesos (5~SPS) trail significantly. The marginal representation of alternatives such as OpenShift (2), Docker Compose (3), and cloud-native services (Amazon ECS/EKS, 1 each) suggests that \textit{academic research has not yet systematically evaluated the trade-offs between Kubernetes and its alternatives}, particularly in edge computing, serverless, and resource-constrained environments where lighter orchestration solutions may be more appropriate.
\begin{center}
    \includegraphics[width=0.5\textwidth]{resources/images/resultados/orch-SPS.png}
    \captionof{figure}{Distribution of orchestrator technologies across SPS}
    \label{fig:SPS-ORCH}
\end{center}

\paragraph{Academic dimension analysis.}
Figure~\ref{fig:SPS-venn} illustrates the intersection of studies across academic dimensions. Research dominates with \textbf{187} exclusive SPS, while Education (19~exclusive) and Outreach (8~exclusive) remain underrepresented. Notably, \textit{no study simultaneously addresses all three dimensions}, revealing a significant fragmentation in academic production. Only 6~SPS bridge Research and Education, and 6~bridge Research and Outreach, with zero overlap between Education and Outreach. This finding suggests that the potential of CBV as a transversal tool linking teaching, research output, and societal impact remains largely unexplored.
\begin{center}
    \includegraphics[width=0.5\textwidth]{resources/images/resultados/SPS-venn.png}
    \captionof{figure}{Venn diagram of SPS across academic dimensions}
    \label{fig:SPS-venn}
\end{center}

\paragraph{Topic and temporal distribution.}
Figure~\ref{fig:SPS-topics} shows topic distribution per research question. For Q1, Research accounts for 83.61\% of studies, while Outreach represents only 5.88\%---a disparity that underscores the limited penetration of containerization into community engagement and societal applications. For Q2, IT Infrastructure leads (41.28\%), followed by Cloud Computing (14.37\%), while Blockchain (0.76\%) and Parallel Computing represent emerging but underexplored intersections with CBV.
\begin{figure}[htbp]
    \centering
    \includegraphics[width=0.5\textwidth]{resources/images/resultados/SPS-topics.png}
    \caption{SPS distribution by research questions and topics}\label{fig:SPS-topics}
\end{figure}

The temporal analysis (Figure~\ref{fig:SPS-QI}) reveals sustained growth, from \textbf{49} SPS in 2022 to \textbf{107} in 2024 (a 118\% cumulative increase). The sharpest growth occurred between 2023 and 2024 (+52.85\%), coinciding with the maturation of Kubernetes-based cloud-native architectures and the proliferation of edge computing applications. The CVI index shows an upward trend (from 7 to 9, +28.57\%), suggesting that more recent studies exhibit stronger alignment with the SMS objectives. The SCI index remains stable around 18, with a slight recovery in 2024 (+11.76\% over 2023), while the IRRQ index exhibits sustained growth from 30 to 39 (+30\% cumulative), indicating increasing thematic breadth in newer publications.

\begin{figure}[htbp]
    \centering
    \includegraphics[width=0.5\textwidth]{resources/images/resultados/SPS-QI.png}
    \caption{SPS by year and quality indices}\label{fig:SPS-QI}
\end{figure}

Figure~\ref{fig:SPS-topics-QI} presents the quality indices disaggregated by topic. IT Infrastructure not only has the highest volume (73~SPS, 43.71\% of Q2) but also concentrates the highest-quality studies across all three indices, reinforcing its centrality to the CBV research landscape. The Outreach topic, with only 8~SPS (7.33\% of Q1), represents the most significant gap identified in this mapping.

Figure~\ref{fig:SPS-kw} presents a cross-analysis of keywords. The term ``\textit{Container}'' enables the identification of 57~SPS, while education-related keywords (\textit{Learning}, \textit{Cybersecurity education}) appear in only 2~SPS each---further evidence that the academic community has not yet developed a robust vocabulary linking CBV to educational and outreach applications.

\onecolumn

\input{figures/study-clasification-by-topic-and-year}
\vspace{1cm}
\begin{center}
\renewcommand{\arraystretch}{1.15}
\begin{longtable}{c Z Y Y Y}
\caption{Studies with the highest CVI index, classified by topics}\label{tab:higher-cvi} \\

\toprule
\textbf{RQ} & \textbf{Topics} & \textbf{2022} & \textbf{2023} & \textbf{2024} \\
\midrule
\endfirsthead

\toprule
\textbf{RQ} & \textbf{Topics} & \textbf{2022} & \textbf{2023} & \textbf{2024} \\
\midrule
\endhead

\multirow{3}{*}{\textbf{Q1}} 
& Research & SPS003, SPS007, SPS083, SPS145, SPS146 & SPS068, SPS174 & SPS032, SPS136, SPS151, SPS168 \\
& Education & SPS038, SPS146 & SPS152, SPS206 & SPS089, SPS115, SPS151 \\
\midrule
\multirow{10}{*}{\textbf{Q2}} 
& Software Development & SPS038 & & \\
& Computational Thinking & & & SPS115 \\
& Data Analysis & SPS037, SPS071, SPS157 & SPS183, SPS209 & SPS001, SPS005, SPS028, SPS045, SPS061, SPS082, SPS129 \\
& IT Infrastructure & SPS003, SPS007, SPS038, SPS083, SPS145, SPS146 & SPS068, SPS152, SPS174, SPS206 & SPS032, SPS089, SPS115, SPS136, SPS151, SPS168 \\
& HPC & SPS083 & & \\
& Security & SPS083 & & \\
& Cloud Computing & SPS003, SPS146 & & SPS032, SPS136 \\
\bottomrule
\end{longtable}
\end{center}

\vspace{10cm}
\begin{center}
\renewcommand{\arraystretch}{1.15}
\begin{longtable}{c Z Y Y Y}
\caption{Studies with the Highest SCI Index, Categorized by Topic}\label{tab:higher-sci} \\

\toprule
\textbf{RQ} & \textbf{Topics} & \textbf{2022} & \textbf{2023} & \textbf{2024} \\
\midrule
\endfirsthead

\toprule
\textbf{RQ} & \textbf{Topics} & \textbf{2022} & \textbf{2023} & \textbf{2024} \\
\midrule
\endhead

\multirow{3}{*}{\textbf{Q1}} 
& Research & SPS003, SPS044, SPS064, SPS083, SPS092, SPS137, SPS143, SPS145, SPS157, SPS176, SPS187, SPS192 & SPS027, SPS029, SPS126, SPS165, SPS173, SPS223 & SPS028, SPS032, SPS033, SPS054, SPS140, SPS197, SPS215 \\
& Education & SPS187 & SPS020, SPS072 & \\
\midrule
\multirow{10}{*}{\textbf{Q2}} 
& Software development & SPS044 & & SPS028, SPS215 \\
& Computational thinking & SPS187 & & \\
& Parallel computing & & SPS020, SPS223 & \\
& Data analysis & SPS157 & & SPS028 \\
& Artificial intelligence & & SPS027, SPS072 & \\
& Computer networks & SPS187 & & \\
& IT infrastructure & SPS003, SPS083, SPS092, SPS137, SPS143, SPS145, SPS176, SPS187 & SPS020, SPS027, SPS029, SPS126, SPS173, SPS223 & SPS032, SPS033, SPS054, SPS140, SPS197, SPS215 \\
& HPC & SPS083 & SPS027 & \\
& Security & SPS064, SPS083, SPS092, SPS157, SPS192 & SPS126, SPS165 & \\
& Cloud computing & SPS003, SPS137, SPS143 & SPS029, SPS126, SPS173 & SPS032, SPS033, SPS197 \\
\bottomrule
\end{longtable}
\end{center}
\vspace{10cm}
\begin{center}
\renewcommand{\arraystretch}{1.15}
\begin{longtable}{c Z Y Y Y}
\caption{Studies with the Highest IRRQ Index, Classified by Topic}\label{tab:higher-irrq} \\

\toprule
\textbf{RQ} & \textbf{Topics} & \textbf{2022} & \textbf{2023} & \textbf{2024} \\
\midrule
\endfirsthead

\toprule
\textbf{RQ} & \textbf{Topics} & \textbf{2022} & \textbf{2023} & \textbf{2024} \\
\midrule
\endhead

\multirow{3}{*}{\textbf{Q1}}
& Research & SPS002, SPS003, SPS007, SPS039, SPS044, SPS053, SPS059, SPS064, SPS070, SPS071, SPS073, SPS080, SPS083, SPS092, SPS137, SPS143, SPS145, SPS146, SPS155, SPS157, SPS176, SPS177, SPS187, SPS192 & SPS027, SPS029, SPS055, SPS066, SPS067, SPS068, SPS081, SPS093, SPS094, SPS117, SPS126, SPS134, SPS153, SPS165, SPS167, SPS173, SPS174, SPS183, SPS195, SPS205, SPS209, SPS221, SPS223, SPS226 & SPS005, SPS008, SPS010, SPS019, SPS021, SPS028, SPS030, SPS032, SPS033, SPS036, SPS045, SPS048, SPS054, SPS061, SPS082, SPS106, SPS107, SPS113, SPS129, SPS136, SPS140, SPS151, SPS168, SPS172, SPS178, SPS184, SPS197, SPS198, SPS214, SPS215, SPS216, SPS219 \\
& Education & SPS038, SPS058, SPS101, SPS146, SPS187, SPS204 & SPS020, SPS072, SPS116, SPS120, SPS152, SPS206, SPS207, SPS218 & SPS089, SPS096, SPS115, SPS151, SPS163, SPS198, SPS199 \\
& Outreach & SPS002, SPS031, SPS037 & SPS078, SPS112 & SPS010, SPS063, SPS114 \\
\midrule
\multirow{10}{*}{\textbf{Q2}}
& Software development & SPS002, SPS037, SPS038, SPS044, SPS053, SPS058, SPS101 & SPS078, SPS120, SPS183, SPS195 & SPS008, SPS010, SPS028, SPS096, SPS172, SPS215 \\
& Computational thinking & SPS187 & SPS116 & SPS115, SPS198 \\
& Parallel computing & & SPS020, SPS134, SPS223 & \\
& Data analysis & SPS037, SPS071, SPS157 & SPS183, SPS209 & SPS005, SPS028, SPS045, SPS061, SPS082, SPS129 \\
& Artificial Intelligence & SPS073 & SPS209 & SPS082 \\
& Computer networks & SPS187 & SPS094 & SPS010, SPS019, SPS048, SPS106, SPS113, SPS198, SPS216, SPS219 \\
& IT infrastructure & SPS003, SPS007, SPS031, SPS037, SPS038, SPS039, SPS070, SPS073, SPS083, SPS092, SPS137, SPS143, SPS145, SPS146, SPS155, SPS176, SPS177, SPS187, SPS204 & SPS020, SPS027, SPS029, SPS055, SPS066, SPS067, SPS068, SPS078, SPS081, SPS094, SPS112, SPS117, SPS126, SPS134, SPS152, SPS167, SPS173, SPS174, SPS183, SPS205, SPS206, SPS207, SPS218, SPS223 & SPS019, SPS021, SPS030, SPS032, SPS033, SPS036, SPS048, SPS054, SPS082, SPS089, SPS096, SPS106, SPS107, SPS115, SPS129, SPS136, SPS140, SPS151, SPS163, SPS168, SPS172, SPS178, SPS184, SPS197, SPS198, SPS199, SPS214, SPS215, SPS216, SPS219 \\
& HPC & SPS083 & SPS027, SPS134 & SPS008, SPS114, SPS129, SPS178 \\
& Security & SPS064, SPS070, SPS083, SPS092, SPS155, SPS157, SPS192 & SPS081, SPS093, SPS094, SPS126, SPS153, SPS165, SPS183, SPS221, SPS226 & SPS082, SPS129, SPS214, SPS219 \\
& Cloud computing & SPS002, SPS003, SPS031, SPS070, SPS071, SPS080, SPS137, SPS143, SPS146, SPS177 & SPS029, SPS055, SPS126, SPS173 & SPS019, SPS030, SPS032, SPS033, SPS045, SPS136, SPS163, SPS197, SPS214, SPS216 \\
\bottomrule
\end{longtable}
\end{center}



\newcolumntype{Z}{>{\centering\arraybackslash}p{3cm}}

\begin{longtable}{Z Y Y Y}
\caption{Classification of SPS studies by technology of \textbf{VBC} and academic dimension}\label{tab:clasificacion-topicos} \\

\toprule
\textbf{Topics} & \textbf{Education} & \textbf{Research} & \textbf{Outreach} \\
\midrule
\endfirsthead

\toprule
\textbf{Topics} & \textbf{Education} & \textbf{Research} & \textbf{Outreach} \\
\midrule
\endhead

CRI-O & & SPS068, SPS083 & \\
\\
Containerd & & SPS066, SPS068, SPS083, SPS223 & \\
\\
Docker & SPS020, SPS038, SPS042, SPS058, SPS072, SPS089, SPS096, SPS101, SPS115, SPS116, SPS120, SPS124, SPS152, SPS187, SPS198, SPS199, SPS204, SPS206, SPS207, SPS218 
& SPS002, SPS004, SPS005, SPS007, SPS008, SPS011, SPS017, SPS021, SPS030, SPS039, SPS040, SPS041, SPS043, SPS044, SPS045, SPS046, SPS048, SPS049, SPS051, SPS053, SPS054, SPS055, SPS059, SPS060, SPS061, SPS065, SPS066, SPS071, SPS074, SPS079, SPS080, SPS081, SPS083, SPS093, SPS097, SPS099, SPS100, SPS102, SPS103, SPS104, SPS105, SPS106, SPS107, SPS119, SPS122, SPS124, SPS126, SPS129, SPS133, SPS153, SPS155, SPS172, SPS173, SPS174, SPS176, SPS177, SPS180, SPS182, SPS187, SPS188, SPS191, SPS192, SPS197, SPS198, SPS205, SPS209, SPS216, SPS219, SPS220, SPS221, SPS225, SPS226 
& SPS002, SPS037, SPS063, SPS078, SPS099, SPS112, SPS114, SPS220 \\
\\
Firecracker & & SPS107, SPS205 & \\
\\
Google gVisor & & SPS107, SPS184, SPS205 & \\
\\
Hyper-V containers & & SPS068 & \\
\\
Kata Containers & & SPS184, SPS205, SPS224 & \\
\\
LXC & & SPS066, SPS068, SPS083, SPS157 & \\
\\
LXD & & SPS068, SPS083 & \\
\\
OpenVZ & & SPS083 & \\
\\
Podman & & SPS007, SPS046, SPS060, SPS068, SPS083, SPS129, SPS174 & \\
\\
Rkt & & SPS068, SPS083 & \\
\\


Singularity & & SPS041, SPS060, SPS068 & \\
\\
Udocker & & SPS027, SPS068 & \\
\bottomrule
\end{longtable}

\begin{longtable}{Z Y Y Y}
\caption{Classification of SPS studies by IT domain and academic dimension}\label{tab:clasificacion-sps} \\

\toprule
\textbf{Topics} & \textbf{Education} & \textbf{Research} & \textbf{Outreach} \\
\midrule
\endfirsthead

\toprule
\textbf{Topics} & \textbf{Education} & \textbf{Research} & \textbf{Outreach} \\
\midrule
\endhead
\addlinespace
Data analysis &  & SPS001, SPS005, SPS028, SPS045, SPS061, SPS071, SPS082, SPS129, SPS157, SPS183, SPS209 & SPS037 \\
\addlinespace
Blockchain &  &  & SPS063 \\
\addlinespace
Cloud computing & SPS146, SPS163 & SPS002, SPS003, SPS012, SPS015, SPS018, SPS019, SPS025, SPS026, SPS029, SPS030, SPS032, SPS033, SPS043, SPS045, SPS055, SPS056, SPS069, SPS070, SPS071, SPS079, SPS080, SPS084, SPS085, SPS087, SPS091, SPS099, SPS109, SPS111, SPS126, SPS136, SPS137, SPS143, SPS146, SPS149, SPS173, SPS177, SPS179, SPS185, SPS193, SPS194, SPS197, SPS202, SPS210, SPS213, SPS214, SPS216, SPS217, SPS222 & SPS002, SPS031, SPS099, SPS213 \\
\addlinespace
Parallel computing & SPS020 & SPS017, SPS134, SPS223 & \\
\addlinespace
Software development & SPS038, SPS042, SPS058, SPS096, SPS101, SPS120 & SPS002, SPS008, SPS010, SPS015, SPS022, SPS028, SPS043, SPS044, SPS053, SPS086, SPS098, SPS100, SPS118, SPS133, SPS172, SPS183, SPS195, SPS215, SPS224 & SPS002, SPS010, SPS037, SPS078 \\
\addlinespace
HPC &  & SPS008, SPS014, SPS017, SPS018, SPS027, SPS041, SPS062, SPS083, SPS090, SPS098, SPS121, SPS129, SPS134, SPS178, SPS194, SPS200 & SPS114 \\
\addlinespace
Artificial intelligence & SPS072 & SPS011, SPS023, SPS027, SPS030, SPS040, SPS051, SPS053, SPS059, SPS073, SPS077, SPS080, SPS082, SPS095, SPS142, SPS148, SPS149, SPS154, SPS161, SPS169, SPS170, SPS177, SPS183, SPS209 & SPS037, SPS078, SPS181 \\
\addlinespace
Computational thinking & SPS042, SPS115, SPS116, SPS187, SPS198 & SPS187, SPS198 & \\
\addlinespace
Computer networks & SPS139, SPS187, SPS198 & SPS010, SPS019, SPS046, SPS048, SPS094, SPS103, SPS105, SPS106, SPS110, SPS113, SPS132, SPS159, SPS164, SPS187, SPS198, SPS216, SPS219 & SPS010 \\
\addlinespace
Security &  & SPS010, SPS019, SPS046, SPS048, SPS094, SPS103, SPS105, SPS106, SPS110, SPS113, SPS132, SPS159, SPS164, SPS187, SPS198, SPS216, SPS219 & \\
\addlinespace
IT infrastructure & SPS020, SPS038, SPS075, SPS089, SPS096, SPS115, SPS124, SPS146, SPS151, SPS152, SPS163, SPS187, SPS198, SPS199, SPS204, SPS206, SPS207, SPS218 & SPS003, SPS004, SPS007, SPS009, SPS011, SPS012, SPS014, SPS017, SPS018, SPS019, SPS021, SPS023, SPS024, SPS025, SPS026, SPS027, SPS029, SPS030, SPS032, SPS033, SPS034, SPS036, SPS039, SPS046, SPS047, SPS048, SPS049, SPS051, SPS052, SPS054, SPS055, SPS056, SPS057, SPS060, SPS062, SPS066, SPS067, SPS068, SPS069, SPS070, SPS073, SPS074, SPS075, SPS076, SPS077, SPS079, SPS081, SPS082, SPS083, SPS084, SPS085, SPS087, SPS088, SPS090, SPS091, SPS092, SPS094, SPS095, SPS099, SPS100, SPS102, SPS103, SPS104, SPS105, SPS106, SPS107, SPS109, SPS110, SPS111, SPS117, SPS119, SPS121, SPS122, SPS123, SPS124, SPS125, SPS126, SPS129, SPS130, SPS131, SPS132, SPS134, SPS135, SPS136, SPS137, SPS140, SPS143, SPS144, SPS145, SPS146, SPS148, SPS149, SPS150 & SPS031, SPS037, SPS078, SPS099, SPS112, SPS181, SPS208, SPS211, SPS213, SPS220 \\
 & & SPS151, SPS154, SPS155, SPS156, SPS159, SPS160, SPS164, SPS167, SPS168, SPS169, SPS170, SPS171, SPS172, SPS173, SPS174, SPS175, SPS176, SPS177, SPS178, SPS179, SPS180, SPS182, SPS183, SPS184, SPS185, SPS186, SPS187, SPS188, SPS189, SPS190, SPS196, SPS197, SPS198, SPS200, SPS201, SPS205, SPS208, SPS210, SPS212, SPS213, SPS214, SPS215, SPS216, SPS217, SPS219, SPS220, SPS222, SPS223, SPS224, SPS225 & \\
\addlinespace
\bottomrule
\end{longtable}

\twocolumn
\begin{figure}[htbp]
    \centering
    \includegraphics[width=0.5\textwidth]{resources/images/resultados/SPS-topics-QI.png}
    \caption{SPS by quality indices, topics, and research questions}\label{fig:SPS-topics-QI}
\end{figure}


\begin{center}
    \includegraphics[scale=0.3]{resources/images/resultados/Kw.png}
    \captionof{figure}{SPS keyword co-occurrence analysis}
    \label{fig:SPS-kw}
\end{center}

\setlength{\tabcolsep}{8pt} % separación horizontal entre columnas

%sub-subseccion 4.6.2
\subsubsection{Word Cloud Visualization}
Figure~\ref{fig:SPS-wordcloud} presents the keyword cloud generated from the 226~SPS (terms with frequency $> 1$). The three dominant clusters---\textit{Docker}, \textit{Container}, \textit{Kubernetes}, \textit{Cloud Computing} (61.6\%); \textit{Containerization}, \textit{Container Orchestration}, \textit{Virtualization}, \textit{Microservices} (13.08\%); and \textit{Performance evaluation}, \textit{Edge computing}, \textit{Machine learning}, \textit{Security} (9.09\%)---reflect the current thematic structure of CBV research. Notably, terms related to education, teaching, and outreach are absent from the high-frequency clusters, confirming the finding that academic applications of CBV remain an under-investigated research area.
\begin{center}
    \includegraphics[width=0.5\textwidth]{resources/images/resultados/wordcloud.png}
    \captionof{figure}{Keyword cloud of the 226 SPS}
    \label{fig:SPS-wordcloud}
\end{center}


% subsection 4.7
\subsubsection{\textit{Reproducibility}}
\label{sec:reproducibility-review-method}

To ensure full reproducibility, two verification mechanisms are provided:

\begin{enumerate}[label=\arabic*)]
  \item A public SMS-Builder instance containing all process data: \href{https://sms-vbc.iti.grid.uniquindio.edu.co/sms.xhtml}{\uline{\textcolor{blue}{https://sms-vbc.iti.grid.uniquindio.edu.co/sms.xhtml}}}.
  Credentials: \textit{``invitado''} for both username and password.

  \item A Docker image integrating all required documentation: \href{https://hub.docker.com/r/anubis1001/tg-vbc-sms-builder}{\uline{\textcolor{blue}{https://hub.docker.com/r/anubis1001/tg-vbc-sms-builder}}}.
\end{enumerate}

%----------------Section 5------------------------
\section{Threats to Validity}
\label{sec:threat-validity}
Four categories of threats to validity are identified, along with the mitigation strategies employed.

\subsection{Selection Bias}
Seven measures were implemented to mitigate selection bias. First, the SMS followed established guidelines~\cite{Runeson2009,Kitchenham2010}, including GQM and PICOC frameworks. Second, five major databases were queried. Third, synonyms were included for all key terms to ensure broad coverage. Fourth, search strings were iteratively refined through pilot searches. Fifth, a hybrid strategy combining database search with snowballing increased coverage. Sixth, alert systems (Endnote, Mendeley, Google Scholar) monitored for newly published studies. Seventh, three quality indices (CVI, SCI, IRRQ) provided complementary assessment perspectives. The CVI and IRRQ indices carry inherent subjectivity; this was mitigated through collaborative evaluation by an odd number of independent evaluators ($K \geq 3$).

\subsection{Classification Errors}
Studies were classified according to the topics defined during planning, corresponding to CBV technologies, IT domains, education, research, and outreach. Multi-topic classification was permitted when a study's scope spanned multiple areas. All classifications underwent peer review by an odd number of evaluators to reduce individual bias.

\subsection{Data Extraction Inaccuracy}
The SMS-Builder software~\cite{candela2020smsbuilder} was used for structured data extraction, minimizing manual processing errors. Peer review was conducted on extracted data following the recommendations of Kitchenham and Charters~\cite{Kitchenham2010}.

\subsection{Search Protocol Errors}
The search protocol was executed under peer review: one evaluator implemented the protocol while a second independently verified the process. SMS-Builder was used throughout to reduce manual data handling and ensure process consistency.


\section{Analysis, Discussion, and Future Research Directions}\label{sec:analysis-and-discussion}

The SMS results reveal several cross-cutting patterns that merit interpretation beyond descriptive statistics. This section synthesizes the key findings, proposes a taxonomic structure, identifies research gaps, and outlines concrete future research directions.

\subsection{Proposed Taxonomic Structure}
The classification of 226~SPS across IT domains and academic dimensions motivates a taxonomic structure (Figure~\ref{fig:taxonomic}) that organizes the CBV research landscape along two axes: \textit{(i)}~the IT domain (software development, IT infrastructure, cloud computing, HPC, security, AI, etc.) and \textit{(ii)}~the academic dimension (education, research, outreach). This dual-axis taxonomy constitutes a novel contribution, as no prior secondary study has simultaneously mapped both dimensions. The taxonomy can serve as a decision-support tool for researchers identifying relevant literature, for educators designing container-based curricula, and for practitioners selecting technologies aligned with domain-specific requirements.

\begin{figure}[htbp]
    \centering
    \includegraphics[width=0.5\textwidth]{resources/images/taxonomy/taxonomia.png}
    \caption{Proposed taxonomic structure for CBV research}
    \label{fig:taxonomic}
\end{figure}

\subsection{Key Findings and Interpretation}

\paragraph{Technology concentration and its implications.}
The dominance of Docker (94~SPS, 41.6\%) and Kubernetes (67~SPS) reveals a significant concentration risk in the research literature. While Docker's ubiquity reflects its first-mover advantage and ecosystem maturity, it also implies that findings from the CBV research corpus may not generalize to alternative runtimes with different security models (gVisor, Kata Containers), permission models (Podman), or HPC optimization (Singularity/Apptainer). Researchers should critically evaluate whether conclusions drawn from Docker-centric studies apply to their specific deployment contexts.

\paragraph{The IT Infrastructure bias.}
IT Infrastructure accounts for 75.66\% of the mapped studies, creating a pronounced bias toward deployment and management concerns. Domains such as Blockchain (0.76\%), Parallel Computing (1.77\%), and Computational Thinking (2.21\%) remain severely underexplored despite clear potential for containerization. For instance, container-based approaches to reproducible blockchain testing environments, portable parallel computing frameworks, and interactive computational thinking platforms represent viable but unaddressed research directions.

\paragraph{The academic dimension gap.}
The fragmentation across academic dimensions---with Research dominating (83.61\%) and Outreach representing only 5.88\%---reveals a missed opportunity. Containerization's core strengths (portability, reproducibility, environment isolation) are precisely the attributes needed for effective outreach and knowledge transfer. The absence of studies simultaneously addressing education, research, and outreach suggests that the academic community has not yet leveraged CBV's transversal potential.

\paragraph{Temporal trends and maturation signals.}
The 118\% growth in publications from 2022 to 2024, combined with the increasing CVI and IRRQ indices, indicates both growing interest and improving methodological alignment with the field's core questions. However, the stable SCI around 18 suggests that while more studies are published, citation impact has not proportionally increased---a pattern consistent with a rapidly expanding but potentially fragmenting research area.

\subsection{Identified Research Gaps}
Based on the SMS findings, the following research gaps are identified:

\begin{enumerate}[label=\textbf{RG\arabic*}:]
    \item \textbf{Alternative runtime evaluation.} Systematic comparative studies of container runtimes beyond Docker (e.g., Podman, Singularity, gVisor, Kata Containers) across performance, security, and usability dimensions are critically needed.
    \item \textbf{Orchestration beyond Kubernetes.} Research evaluating lightweight orchestration alternatives (K3s, Nomad, Docker Compose) for edge computing, IoT, and resource-constrained environments is underrepresented.
    \item \textbf{CBV in education and outreach.} Empirical studies measuring the impact of containerization on learning outcomes, curriculum design, and outreach program effectiveness are nearly absent from the literature.
    \item \textbf{Cross-domain integration.} No identified study bridges all three academic dimensions simultaneously, presenting an opportunity for holistic CBV adoption frameworks.
    \item \textbf{Emerging IT domains.} The intersection of CBV with blockchain, quantum computing simulation, and computational thinking requires dedicated investigation.
    \item \textbf{Security of container ecosystems.} While security appears in 65~SPS, most address container isolation rather than supply-chain security, image provenance, or runtime attestation---areas of growing practical concern.
\end{enumerate}

\subsection{Future Research Directions}
Building on the identified gaps, the following concrete research directions are proposed:

\begin{itemize}
    \item \textbf{Benchmarking frameworks:} Development of standardized benchmarking methodologies for comparing container runtimes and orchestrators across heterogeneous hardware (x86, ARM, RISC-V) and workload profiles (HPC, microservices, AI training).
    \item \textbf{Educational impact studies:} Controlled experiments measuring the effect of container-based laboratory environments on student learning outcomes, engagement, and skill transferability in computer science education.
    \item \textbf{Outreach and knowledge transfer:} Design and evaluation of container-packaged educational platforms that facilitate technology transfer from universities to industry and communities, particularly in resource-limited settings.
    \item \textbf{Security posture analysis:} Comprehensive studies on container supply-chain security, including image scanning effectiveness, Software Bill of Materials (SBOM) adoption, and runtime security monitoring in production environments.
    \item \textbf{Lightweight orchestration for edge/IoT:} Empirical evaluation of Kubernetes alternatives in edge and IoT deployments where resource constraints and latency requirements differ from cloud-native assumptions.
\end{itemize}

\subsection{Implications for Research and Practice}
For \textit{researchers}, this SMS provides a structured entry point into the CBV literature, enabling targeted investigation of underexplored domains and informed positioning of new contributions. The identified gaps (RG1--RG6) offer concrete starting points for future studies. For \textit{practitioners}, the technology distribution analysis highlights both the safety of Docker/Kubernetes adoption (given extensive research backing) and the risk of overlooking better-suited alternatives for specific use cases. For \textit{educators}, the near-absence of containerization in formal educational frameworks represents an opportunity to develop innovative pedagogical approaches leveraging CBV's reproducibility and portability.

\section{Conclusions}
\label{sec:conclusions}
This paper presented a Systematic Mapping Study (SMS) on container-based virtualization (CBV) technologies, covering 226 primary studies published between 2022 and 2024. The study employed a hybrid search strategy (database queries and snowballing), three quality assessment indices (CVI, SCI, IRRQ), and a dual-axis classification spanning 11 IT domains and three academic dimensions (education, research, outreach).

The main contributions and findings are as follows:

\begin{enumerate}
    \item \textbf{Comprehensive mapping:} The SMS provides the first systematic mapping that simultaneously covers CBV technologies across IT domains and academic dimensions, addressing a gap identified in all prior secondary studies.
    \item \textbf{Technology concentration:} Docker (41.6\%) and Kubernetes (29.6\%) dominate the research landscape, while alternative runtimes (Podman, Singularity, gVisor) and orchestrators (K3s, Nomad) remain significantly underrepresented despite their growing industrial relevance.
    \item \textbf{Domain imbalance:} IT Infrastructure accounts for 75.66\% of the corpus, while domains such as Blockchain (0.76\%), Parallel Computing (1.77\%), and Computational Thinking (2.21\%) represent critical blind spots in current research.
    \item \textbf{Academic fragmentation:} Research dominates (83.61\%) while Education (11.06\%) and Outreach (5.88\%) remain underexplored. No study simultaneously addresses all three academic dimensions, indicating that CBV's transversal potential for academic activities remains unrealized.
    \item \textbf{Taxonomic structure:} A novel dual-axis taxonomy organizes the CBV literature by IT domain and academic dimension, providing a structured reference for researchers, educators, and practitioners.
    \item \textbf{Research gaps:} Six concrete gaps (RG1--RG6) were identified, with corresponding future research directions including alternative runtime benchmarking, educational impact studies, outreach frameworks, supply-chain security analysis, and lightweight orchestration for edge/IoT.
\end{enumerate}

The growing publication rate (118\% increase from 2022 to 2024) confirms the vitality of this research area but also underscores the risk of literature saturation without adequate systematization. The taxonomic structure and reproducibility artifacts provided with this study aim to mitigate this challenge.

As future work, we plan to conduct controlled comparative evaluations of container runtimes and orchestrators across heterogeneous hardware platforms, and to design and evaluate container-based pedagogical frameworks for computer science education.


% Appendices
\appendix

\onecolumn
\section{Complete List of Selected Primary Studies}\label{app:sps-list}
Table~\ref{tab:sps-list} provides the complete list of the 226 selected primary studies (SPS) included in this SMS.

\begin{longtable*}{c>{\centering\arraybackslash}p{6cm}
                 c>{\centering\arraybackslash}p{6cm}}
\label{tab:sps-list} \\
\textbf{ID} & \textbf{Ref} &
\textbf{ID} & \textbf{Ref} \\
\hline
\endfirsthead

\multicolumn{4}{c}{{\bfseries \tablename\ \thetable{} -- continued}} \\
\textbf{ID} & \textbf{Ref} &
\textbf{ID} & \textbf{Ref} \\
\hline
\endhead

\hline \multicolumn{4}{r}{{Continues on the next page}} \\
\endfoot

\hline
\endlastfoot
SPS001 & \citeauthor{1} & SPS002 & \citeauthor{2} \\
SPS003 & \citeauthor{3} & SPS004 & \citeauthor{4} \\
SPS005 & \citeauthor{5} & SPS006 & \citeauthor{6} \\
SPS007 & \citeauthor{7} & SPS008 & \citeauthor{8} \\
SPS009 & \citeauthor{9} & SPS010 & \citeauthor{10} \\
SPS011 & \citeauthor{11} & SPS012 & \citeauthor{12} \\
SPS013 & \citeauthor{13} & SPS014 & \citeauthor{14} \\
SPS015 & \citeauthor{15} & SPS016 & \citeauthor{16} \\
SPS017 & \citeauthor{17} & SPS018 & \citeauthor{18} \\
SPS019 & \citeauthor{19} & SPS020 & \citeauthor{20} \\
SPS021 & \citeauthor{21} & SPS022 & \citeauthor{22} \\
SPS023 & \citeauthor{23} & SPS024 & \citeauthor{24} \\
SPS025 & \citeauthor{25} & SPS026 & \citeauthor{26} \\
SPS027 & \citeauthor{27} & SPS028 & \citeauthor{28} \\
SPS029 & \citeauthor{29} & SPS030 & \citeauthor{30} \\
SPS031 & \citeauthor{31} & SPS032 & \citeauthor{32} \\
SPS033 & \citeauthor{33} & SPS034 & \citeauthor{34} \\
SPS035 & \citeauthor{35} & SPS036 & \citeauthor{36} \\
SPS037 & \citeauthor{37} & SPS038 & \citeauthor{38} \\
SPS039 & \citeauthor{39} & SPS040 & \citeauthor{40} \\
SPS041 & \citeauthor{41} & SPS042 & \citeauthor{42} \\
SPS043 & \citeauthor{43} & SPS044 & \citeauthor{44} \\
SPS045 & \citeauthor{45} & SPS046 & \citeauthor{46} \\
SPS047 & \citeauthor{47} & SPS048 & \citeauthor{48} \\
SPS049 & \citeauthor{49} & SPS050 & \citeauthor{50} \\
SPS051 & \citeauthor{51} & SPS052 & \citeauthor{52} \\
SPS053 & \citeauthor{53} & SPS054 & \citeauthor{54} \\
SPS055 & \citeauthor{55} & SPS056 & \citeauthor{56} \\
SPS057 & \citeauthor{57} & SPS058 & \citeauthor{58} \\
SPS059 & \citeauthor{59} & SPS060 & \citeauthor{60} \\
SPS061 & \citeauthor{61} & SPS062 & \citeauthor{62} \\
SPS063 & \citeauthor{63} & SPS064 & \citeauthor{64} \\
SPS065 & \citeauthor{65} & SPS066 & \citeauthor{66} \\
SPS067 & \citeauthor{67} & SPS068 & \citeauthor{68} \\
SPS069 & \citeauthor{69} & SPS070 & \citeauthor{70} \\
SPS071 & \citeauthor{71} & SPS072 & \citeauthor{72} \\
SPS073 & \citeauthor{73} & SPS074 & \citeauthor{74} \\
SPS075 & \citeauthor{75} & SPS076 & \citeauthor{76} \\
SPS077 & \citeauthor{77} & SPS078 & \citeauthor{78} \\
SPS079 & \citeauthor{79} & SPS080 & \citeauthor{80} \\
SPS081 & \citeauthor{81} & SPS082 & \citeauthor{82} \\
SPS083 & \citeauthor{83} & SPS084 & \citeauthor{84} \\
SPS085 & \citeauthor{85} & SPS086 & \citeauthor{86} \\
SPS087 & \citeauthor{87} & SPS088 & \citeauthor{88} \\
SPS089 & \citeauthor{89} & SPS090 & \citeauthor{90} \\
SPS091 & \citeauthor{91} & SPS092 & \citeauthor{92} \\
SPS093 & \citeauthor{93} & SPS094 & \citeauthor{94} \\
SPS095 & \citeauthor{95} & SPS096 & \citeauthor{96} \\
SPS097 & \citeauthor{97} & SPS098 & \citeauthor{98} \\
SPS099 & \citeauthor{99} & SPS100 & \citeauthor{100} \\
SPS101 & \citeauthor{101} & SPS102 & \citeauthor{102} \\
SPS103 & \citeauthor{103} & SPS104 & \citeauthor{104} \\
SPS105 & \citeauthor{105} & SPS106 & \citeauthor{106} \\
SPS107 & \citeauthor{107} & SPS108 & \citeauthor{108} \\
SPS109 & \citeauthor{109} & SPS110 & \citeauthor{110} \\
SPS111 & \citeauthor{111} & SPS112 & \citeauthor{112} \\
SPS113 & \citeauthor{113} & SPS114 & \citeauthor{114} \\
SPS115 & \citeauthor{115} & SPS116 & \citeauthor{116} \\
SPS117 & \citeauthor{117} & SPS118 & \citeauthor{118} \\
SPS119 & \citeauthor{119} & SPS120 & \citeauthor{120} \\
SPS121 & \citeauthor{121} & SPS122 & \citeauthor{122} \\
SPS123 & \citeauthor{123} & SPS124 & \citeauthor{124} \\
SPS125 & \citeauthor{125} & SPS126 & \citeauthor{126} \\
SPS127 & \citeauthor{127} & SPS128 & \citeauthor{128} \\
SPS129 & \citeauthor{129} & SPS130 & \citeauthor{130} \\
SPS131 & \citeauthor{131} & SPS132 & \citeauthor{132} \\
SPS133 & \citeauthor{133} & SPS134 & \citeauthor{134} \\
SPS135 & \citeauthor{135} & SPS136 & \citeauthor{136} \\
SPS137 & \citeauthor{137} & SPS138 & \citeauthor{138} \\
SPS139 & \citeauthor{139} & SPS140 & \citeauthor{140} \\
SPS141 & \citeauthor{141} & SPS142 & \citeauthor{142} \\
SPS143 & \citeauthor{143} & SPS144 & \citeauthor{144} \\
SPS145 & \citeauthor{145} & SPS146 & \citeauthor{146} \\
SPS147 & \citeauthor{147} & SPS148 & \citeauthor{148} \\
SPS149 & \citeauthor{149} & SPS150 & \citeauthor{150} \\
SPS151 & \citeauthor{151} & SPS152 & \citeauthor{152} \\
SPS153 & \citeauthor{153} & SPS154 & \citeauthor{154} \\
SPS155 & \citeauthor{155} & SPS156 & \citeauthor{156} \\
SPS157 & \citeauthor{157} & SPS158 & \citeauthor{158} \\
SPS159 & \citeauthor{159} & SPS160 & \citeauthor{160} \\
SPS161 & \citeauthor{161} & SPS162 & \citeauthor{162} \\
SPS163 & \citeauthor{163} & SPS164 & \citeauthor{164} \\
SPS165 & \citeauthor{165} & SPS166 & \citeauthor{166} \\
SPS167 & \citeauthor{167} & SPS168 & \citeauthor{168} \\
SPS169 & \citeauthor{169} & SPS170 & \citeauthor{170} \\
SPS171 & \citeauthor{171} & SPS172 & \citeauthor{172} \\
SPS173 & \citeauthor{173} & SPS174 & \citeauthor{174} \\
SPS175 & \citeauthor{175} & SPS176 & \citeauthor{176} \\
SPS177 & \citeauthor{177} & SPS178 & \citeauthor{178} \\
SPS179 & \citeauthor{179} & SPS180 & \citeauthor{180} \\
SPS181 & \citeauthor{181} & SPS182 & \citeauthor{182} \\
SPS183 & \citeauthor{183} & SPS184 & \citeauthor{184} \\
SPS185 & \citeauthor{185} & SPS186 & \citeauthor{186} \\
SPS187 & \citeauthor{187} & SPS188 & \citeauthor{188} \\
SPS189 & \citeauthor{189} & SPS190 & \citeauthor{190} \\
SPS191 & \citeauthor{191} & SPS192 & \citeauthor{192} \\
SPS193 & \citeauthor{193} & SPS194 & \citeauthor{194} \\
SPS195 & \citeauthor{195} & SPS196 & \citeauthor{196} \\
SPS197 & \citeauthor{197} & SPS198 & \citeauthor{198} \\
SPS199 & \citeauthor{199} & SPS200 & \citeauthor{200} \\
SPS201 & \citeauthor{201} & SPS202 & \citeauthor{202} \\
SPS203 & \citeauthor{203} & SPS204 & \citeauthor{204} \\
SPS205 & \citeauthor{205} & SPS206 & \citeauthor{206} \\
SPS207 & \citeauthor{207} & SPS208 & \citeauthor{208} \\
SPS209 & \citeauthor{209} & SPS210 & \citeauthor{210} \\
SPS211 & \citeauthor{211} & SPS212 & \citeauthor{212} \\
SPS213 & \citeauthor{213} & SPS214 & \citeauthor{214} \\
SPS215 & \citeauthor{215} & SPS216 & \citeauthor{216} \\
SPS217 & \citeauthor{217} & SPS218 & \citeauthor{218} \\
SPS219 & \citeauthor{219} & SPS220 & \citeauthor{220} \\
SPS221 & \citeauthor{221} & SPS222 & \citeauthor{222} \\
SPS223 & \citeauthor{223} & SPS224 & \citeauthor{224} \\
SPS225 & \citeauthor{225} & SPS226 & \citeauthor{226} \\
\end{longtable*}

\section{Detailed Classification Tables}\label{app:classification-tables}

Table~\ref{tab:clasificacion} presents the full classification of SPS studies by topic and year.

\input{figures/study-clasification-by-topic-and-year}

\section{Quality Index Tables}\label{app:quality-index-tables}

Tables~\ref{tab:higher-cvi}, \ref{tab:higher-sci}, and \ref{tab:higher-irrq} present the studies with the highest quality index scores, disaggregated by topic and year.

\begin{center}
\renewcommand{\arraystretch}{1.15}
\begin{longtable}{c Z Y Y Y}
\caption{Studies with the highest CVI index, classified by topics}\label{tab:higher-cvi} \\

\toprule
\textbf{RQ} & \textbf{Topics} & \textbf{2022} & \textbf{2023} & \textbf{2024} \\
\midrule
\endfirsthead

\toprule
\textbf{RQ} & \textbf{Topics} & \textbf{2022} & \textbf{2023} & \textbf{2024} \\
\midrule
\endhead

\multirow{3}{*}{\textbf{Q1}} 
& Research & SPS003, SPS007, SPS083, SPS145, SPS146 & SPS068, SPS174 & SPS032, SPS136, SPS151, SPS168 \\
& Education & SPS038, SPS146 & SPS152, SPS206 & SPS089, SPS115, SPS151 \\
\midrule
\multirow{10}{*}{\textbf{Q2}} 
& Software Development & SPS038 & & \\
& Computational Thinking & & & SPS115 \\
& Data Analysis & SPS037, SPS071, SPS157 & SPS183, SPS209 & SPS001, SPS005, SPS028, SPS045, SPS061, SPS082, SPS129 \\
& IT Infrastructure & SPS003, SPS007, SPS038, SPS083, SPS145, SPS146 & SPS068, SPS152, SPS174, SPS206 & SPS032, SPS089, SPS115, SPS136, SPS151, SPS168 \\
& HPC & SPS083 & & \\
& Security & SPS083 & & \\
& Cloud Computing & SPS003, SPS146 & & SPS032, SPS136 \\
\bottomrule
\end{longtable}
\end{center}


\begin{center}
\renewcommand{\arraystretch}{1.15}
\begin{longtable}{c Z Y Y Y}
\caption{Studies with the Highest SCI Index, Categorized by Topic}\label{tab:higher-sci} \\

\toprule
\textbf{RQ} & \textbf{Topics} & \textbf{2022} & \textbf{2023} & \textbf{2024} \\
\midrule
\endfirsthead

\toprule
\textbf{RQ} & \textbf{Topics} & \textbf{2022} & \textbf{2023} & \textbf{2024} \\
\midrule
\endhead

\multirow{3}{*}{\textbf{Q1}} 
& Research & SPS003, SPS044, SPS064, SPS083, SPS092, SPS137, SPS143, SPS145, SPS157, SPS176, SPS187, SPS192 & SPS027, SPS029, SPS126, SPS165, SPS173, SPS223 & SPS028, SPS032, SPS033, SPS054, SPS140, SPS197, SPS215 \\
& Education & SPS187 & SPS020, SPS072 & \\
\midrule
\multirow{10}{*}{\textbf{Q2}} 
& Software development & SPS044 & & SPS028, SPS215 \\
& Computational thinking & SPS187 & & \\
& Parallel computing & & SPS020, SPS223 & \\
& Data analysis & SPS157 & & SPS028 \\
& Artificial intelligence & & SPS027, SPS072 & \\
& Computer networks & SPS187 & & \\
& IT infrastructure & SPS003, SPS083, SPS092, SPS137, SPS143, SPS145, SPS176, SPS187 & SPS020, SPS027, SPS029, SPS126, SPS173, SPS223 & SPS032, SPS033, SPS054, SPS140, SPS197, SPS215 \\
& HPC & SPS083 & SPS027 & \\
& Security & SPS064, SPS083, SPS092, SPS157, SPS192 & SPS126, SPS165 & \\
& Cloud computing & SPS003, SPS137, SPS143 & SPS029, SPS126, SPS173 & SPS032, SPS033, SPS197 \\
\bottomrule
\end{longtable}
\end{center}

\begin{center}
\renewcommand{\arraystretch}{1.15}
\begin{longtable}{c Z Y Y Y}
\caption{Studies with the Highest IRRQ Index, Classified by Topic}\label{tab:higher-irrq} \\

\toprule
\textbf{RQ} & \textbf{Topics} & \textbf{2022} & \textbf{2023} & \textbf{2024} \\
\midrule
\endfirsthead

\toprule
\textbf{RQ} & \textbf{Topics} & \textbf{2022} & \textbf{2023} & \textbf{2024} \\
\midrule
\endhead

\multirow{3}{*}{\textbf{Q1}}
& Research & SPS002, SPS003, SPS007, SPS039, SPS044, SPS053, SPS059, SPS064, SPS070, SPS071, SPS073, SPS080, SPS083, SPS092, SPS137, SPS143, SPS145, SPS146, SPS155, SPS157, SPS176, SPS177, SPS187, SPS192 & SPS027, SPS029, SPS055, SPS066, SPS067, SPS068, SPS081, SPS093, SPS094, SPS117, SPS126, SPS134, SPS153, SPS165, SPS167, SPS173, SPS174, SPS183, SPS195, SPS205, SPS209, SPS221, SPS223, SPS226 & SPS005, SPS008, SPS010, SPS019, SPS021, SPS028, SPS030, SPS032, SPS033, SPS036, SPS045, SPS048, SPS054, SPS061, SPS082, SPS106, SPS107, SPS113, SPS129, SPS136, SPS140, SPS151, SPS168, SPS172, SPS178, SPS184, SPS197, SPS198, SPS214, SPS215, SPS216, SPS219 \\
& Education & SPS038, SPS058, SPS101, SPS146, SPS187, SPS204 & SPS020, SPS072, SPS116, SPS120, SPS152, SPS206, SPS207, SPS218 & SPS089, SPS096, SPS115, SPS151, SPS163, SPS198, SPS199 \\
& Outreach & SPS002, SPS031, SPS037 & SPS078, SPS112 & SPS010, SPS063, SPS114 \\
\midrule
\multirow{10}{*}{\textbf{Q2}}
& Software development & SPS002, SPS037, SPS038, SPS044, SPS053, SPS058, SPS101 & SPS078, SPS120, SPS183, SPS195 & SPS008, SPS010, SPS028, SPS096, SPS172, SPS215 \\
& Computational thinking & SPS187 & SPS116 & SPS115, SPS198 \\
& Parallel computing & & SPS020, SPS134, SPS223 & \\
& Data analysis & SPS037, SPS071, SPS157 & SPS183, SPS209 & SPS005, SPS028, SPS045, SPS061, SPS082, SPS129 \\
& Artificial Intelligence & SPS073 & SPS209 & SPS082 \\
& Computer networks & SPS187 & SPS094 & SPS010, SPS019, SPS048, SPS106, SPS113, SPS198, SPS216, SPS219 \\
& IT infrastructure & SPS003, SPS007, SPS031, SPS037, SPS038, SPS039, SPS070, SPS073, SPS083, SPS092, SPS137, SPS143, SPS145, SPS146, SPS155, SPS176, SPS177, SPS187, SPS204 & SPS020, SPS027, SPS029, SPS055, SPS066, SPS067, SPS068, SPS078, SPS081, SPS094, SPS112, SPS117, SPS126, SPS134, SPS152, SPS167, SPS173, SPS174, SPS183, SPS205, SPS206, SPS207, SPS218, SPS223 & SPS019, SPS021, SPS030, SPS032, SPS033, SPS036, SPS048, SPS054, SPS082, SPS089, SPS096, SPS106, SPS107, SPS115, SPS129, SPS136, SPS140, SPS151, SPS163, SPS168, SPS172, SPS178, SPS184, SPS197, SPS198, SPS199, SPS214, SPS215, SPS216, SPS219 \\
& HPC & SPS083 & SPS027, SPS134 & SPS008, SPS114, SPS129, SPS178 \\
& Security & SPS064, SPS070, SPS083, SPS092, SPS155, SPS157, SPS192 & SPS081, SPS093, SPS094, SPS126, SPS153, SPS165, SPS183, SPS221, SPS226 & SPS082, SPS129, SPS214, SPS219 \\
& Cloud computing & SPS002, SPS003, SPS031, SPS070, SPS071, SPS080, SPS137, SPS143, SPS146, SPS177 & SPS029, SPS055, SPS126, SPS173 & SPS019, SPS030, SPS032, SPS033, SPS045, SPS136, SPS163, SPS197, SPS214, SPS216 \\
\bottomrule
\end{longtable}
\end{center}

\section{Technology and Domain Classification Tables}\label{app:technology-domain-tables}

Table~\ref{tab:clasificacion-topicos} classifies SPS studies by CBV technology and academic dimension. Table~\ref{tab:clasificacion-sps} classifies SPS studies by IT domain and academic dimension.



\newcolumntype{Z}{>{\centering\arraybackslash}p{3cm}}

\begin{longtable}{Z Y Y Y}
\caption{Classification of SPS studies by technology of \textbf{VBC} and academic dimension}\label{tab:clasificacion-topicos} \\

\toprule
\textbf{Topics} & \textbf{Education} & \textbf{Research} & \textbf{Outreach} \\
\midrule
\endfirsthead

\toprule
\textbf{Topics} & \textbf{Education} & \textbf{Research} & \textbf{Outreach} \\
\midrule
\endhead

CRI-O & & SPS068, SPS083 & \\
\\
Containerd & & SPS066, SPS068, SPS083, SPS223 & \\
\\
Docker & SPS020, SPS038, SPS042, SPS058, SPS072, SPS089, SPS096, SPS101, SPS115, SPS116, SPS120, SPS124, SPS152, SPS187, SPS198, SPS199, SPS204, SPS206, SPS207, SPS218 
& SPS002, SPS004, SPS005, SPS007, SPS008, SPS011, SPS017, SPS021, SPS030, SPS039, SPS040, SPS041, SPS043, SPS044, SPS045, SPS046, SPS048, SPS049, SPS051, SPS053, SPS054, SPS055, SPS059, SPS060, SPS061, SPS065, SPS066, SPS071, SPS074, SPS079, SPS080, SPS081, SPS083, SPS093, SPS097, SPS099, SPS100, SPS102, SPS103, SPS104, SPS105, SPS106, SPS107, SPS119, SPS122, SPS124, SPS126, SPS129, SPS133, SPS153, SPS155, SPS172, SPS173, SPS174, SPS176, SPS177, SPS180, SPS182, SPS187, SPS188, SPS191, SPS192, SPS197, SPS198, SPS205, SPS209, SPS216, SPS219, SPS220, SPS221, SPS225, SPS226 
& SPS002, SPS037, SPS063, SPS078, SPS099, SPS112, SPS114, SPS220 \\
\\
Firecracker & & SPS107, SPS205 & \\
\\
Google gVisor & & SPS107, SPS184, SPS205 & \\
\\
Hyper-V containers & & SPS068 & \\
\\
Kata Containers & & SPS184, SPS205, SPS224 & \\
\\
LXC & & SPS066, SPS068, SPS083, SPS157 & \\
\\
LXD & & SPS068, SPS083 & \\
\\
OpenVZ & & SPS083 & \\
\\
Podman & & SPS007, SPS046, SPS060, SPS068, SPS083, SPS129, SPS174 & \\
\\
Rkt & & SPS068, SPS083 & \\
\\


Singularity & & SPS041, SPS060, SPS068 & \\
\\
Udocker & & SPS027, SPS068 & \\
\bottomrule
\end{longtable}

\begin{longtable}{Z Y Y Y}
\caption{Classification of SPS studies by IT domain and academic dimension}\label{tab:clasificacion-sps} \\

\toprule
\textbf{Topics} & \textbf{Education} & \textbf{Research} & \textbf{Outreach} \\
\midrule
\endfirsthead

\toprule
\textbf{Topics} & \textbf{Education} & \textbf{Research} & \textbf{Outreach} \\
\midrule
\endhead
\addlinespace
Data analysis &  & SPS001, SPS005, SPS028, SPS045, SPS061, SPS071, SPS082, SPS129, SPS157, SPS183, SPS209 & SPS037 \\
\addlinespace
Blockchain &  &  & SPS063 \\
\addlinespace
Cloud computing & SPS146, SPS163 & SPS002, SPS003, SPS012, SPS015, SPS018, SPS019, SPS025, SPS026, SPS029, SPS030, SPS032, SPS033, SPS043, SPS045, SPS055, SPS056, SPS069, SPS070, SPS071, SPS079, SPS080, SPS084, SPS085, SPS087, SPS091, SPS099, SPS109, SPS111, SPS126, SPS136, SPS137, SPS143, SPS146, SPS149, SPS173, SPS177, SPS179, SPS185, SPS193, SPS194, SPS197, SPS202, SPS210, SPS213, SPS214, SPS216, SPS217, SPS222 & SPS002, SPS031, SPS099, SPS213 \\
\addlinespace
Parallel computing & SPS020 & SPS017, SPS134, SPS223 & \\
\addlinespace
Software development & SPS038, SPS042, SPS058, SPS096, SPS101, SPS120 & SPS002, SPS008, SPS010, SPS015, SPS022, SPS028, SPS043, SPS044, SPS053, SPS086, SPS098, SPS100, SPS118, SPS133, SPS172, SPS183, SPS195, SPS215, SPS224 & SPS002, SPS010, SPS037, SPS078 \\
\addlinespace
HPC &  & SPS008, SPS014, SPS017, SPS018, SPS027, SPS041, SPS062, SPS083, SPS090, SPS098, SPS121, SPS129, SPS134, SPS178, SPS194, SPS200 & SPS114 \\
\addlinespace
Artificial intelligence & SPS072 & SPS011, SPS023, SPS027, SPS030, SPS040, SPS051, SPS053, SPS059, SPS073, SPS077, SPS080, SPS082, SPS095, SPS142, SPS148, SPS149, SPS154, SPS161, SPS169, SPS170, SPS177, SPS183, SPS209 & SPS037, SPS078, SPS181 \\
\addlinespace
Computational thinking & SPS042, SPS115, SPS116, SPS187, SPS198 & SPS187, SPS198 & \\
\addlinespace
Computer networks & SPS139, SPS187, SPS198 & SPS010, SPS019, SPS046, SPS048, SPS094, SPS103, SPS105, SPS106, SPS110, SPS113, SPS132, SPS159, SPS164, SPS187, SPS198, SPS216, SPS219 & SPS010 \\
\addlinespace
Security &  & SPS010, SPS019, SPS046, SPS048, SPS094, SPS103, SPS105, SPS106, SPS110, SPS113, SPS132, SPS159, SPS164, SPS187, SPS198, SPS216, SPS219 & \\
\addlinespace
IT infrastructure & SPS020, SPS038, SPS075, SPS089, SPS096, SPS115, SPS124, SPS146, SPS151, SPS152, SPS163, SPS187, SPS198, SPS199, SPS204, SPS206, SPS207, SPS218 & SPS003, SPS004, SPS007, SPS009, SPS011, SPS012, SPS014, SPS017, SPS018, SPS019, SPS021, SPS023, SPS024, SPS025, SPS026, SPS027, SPS029, SPS030, SPS032, SPS033, SPS034, SPS036, SPS039, SPS046, SPS047, SPS048, SPS049, SPS051, SPS052, SPS054, SPS055, SPS056, SPS057, SPS060, SPS062, SPS066, SPS067, SPS068, SPS069, SPS070, SPS073, SPS074, SPS075, SPS076, SPS077, SPS079, SPS081, SPS082, SPS083, SPS084, SPS085, SPS087, SPS088, SPS090, SPS091, SPS092, SPS094, SPS095, SPS099, SPS100, SPS102, SPS103, SPS104, SPS105, SPS106, SPS107, SPS109, SPS110, SPS111, SPS117, SPS119, SPS121, SPS122, SPS123, SPS124, SPS125, SPS126, SPS129, SPS130, SPS131, SPS132, SPS134, SPS135, SPS136, SPS137, SPS140, SPS143, SPS144, SPS145, SPS146, SPS148, SPS149, SPS150 & SPS031, SPS037, SPS078, SPS099, SPS112, SPS181, SPS208, SPS211, SPS213, SPS220 \\
 & & SPS151, SPS154, SPS155, SPS156, SPS159, SPS160, SPS164, SPS167, SPS168, SPS169, SPS170, SPS171, SPS172, SPS173, SPS174, SPS175, SPS176, SPS177, SPS178, SPS179, SPS180, SPS182, SPS183, SPS184, SPS185, SPS186, SPS187, SPS188, SPS189, SPS190, SPS196, SPS197, SPS198, SPS200, SPS201, SPS205, SPS208, SPS210, SPS212, SPS213, SPS214, SPS215, SPS216, SPS217, SPS219, SPS220, SPS222, SPS223, SPS224, SPS225 & \\
\addlinespace
\bottomrule
\end{longtable}
\twocolumn


\bibliographystyle{plainnat}
\bibliography{refs}

\end{document}
