%-----------Section 2---------------
\section{Motivation}
\label{sec:motivation}
The adoption of cloud computing has produced a wide range of solutions based on container-based virtualization~\cite{HASSAN2022100138}. However, as noted by multiple authors~\cite{waseem2024containerization,VhatkarBhole2022_Survey_ContainerResourceAllocation,kithulwatta2022integration}, the literature remains fragmented: the high volume of publications makes it difficult to identify clear usage patterns, benefits, and limitations across application domains.

This study is motivated by three specific needs. First, there is no existing secondary study that simultaneously maps CBV technologies across multiple IT domains \textit{and} academic dimensions (education, research, outreach). Second, the rapid proliferation of container technologies beyond Docker---including Podman, Singularity, gVisor, and others---requires a systematic assessment of their adoption and research coverage. Third, the potential of CBV as a transversal tool for academic activities (reproducible research, portable teaching environments, accessible outreach platforms) has not been systematically evaluated.

The expected outcomes include: \textit{(i)}~a comprehensive map of CBV research trends across 11 IT domains; \textit{(ii)}~a classification of studies by their contribution to education, research, and outreach; and \textit{(iii)}~a taxonomic structure that can guide technological decision-making for researchers, educators, and practitioners.