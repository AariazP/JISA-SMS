%-----------Section 1---------------
\section{Introduction}\label{sec:intro}
Cloud computing has become a widely adopted paradigm in contemporary information technology, enabling scalable and resilient solutions through on-demand resource provisioning~\citep{falade-segun}. Among the foundational technologies supporting this paradigm, container-based virtualization (CBV) has gained wide adoption due to its lightweight footprint, portability, and rapid deployment capabilities~\citep{mohamed-almoudane}. Unlike full virtualization, which emulates complete hardware environments, containers share the host operating system kernel while maintaining application isolation, resulting in substantially lower overhead~\citep{KOZHIRBAYEV2017175}. These properties have positioned CBV as a commonly used approach for deploying, managing, and scaling applications across distributed environments, facilitating continuous integration, agile development, and microservices architectures~\citep{clemente-mateo}.

Within the CBV ecosystem, Docker has emerged as the \textit{de facto} standard for container creation and management, while Kubernetes dominates container orchestration. However, the rapid evolution of this field has produced a diverse landscape of alternative technologies---including Podman, Singularity, LXC, gVisor, and Kata Containers---each optimized for specific use cases such as rootless operation, high-performance computing, or enhanced security~\citep{BARESI2024111965}. This technological diversification necessitates a systematic analysis to identify which technologies are adopted in which contexts and to what extent.

Beyond industrial applications, CBV technologies have increasingly permeated academic settings, supporting education through reproducible laboratory environments, enabling research through portable computational pipelines, and facilitating outreach through accessible cloud-based platforms. However, the extent and nature of this academic adoption remain poorly characterized in the literature.

This paper presents a Systematic Mapping Study (SMS) that addresses this gap by mapping 226 primary studies (2022--2024) across 11 IT domains and three academic dimensions (education, research, outreach). The study contributes: \textit{(1)}~a comprehensive, reproducible mapping of the CBV research landscape; \textit{(2)}~a proposed dual-axis taxonomic structure linking technologies to both IT domains and academic impact; and \textit{(3)}~the identification of six concrete research gaps with actionable future directions.

The remainder of this paper is organized as follows: Section~\ref{sec:motivation} outlines the study motivation. Section~\ref{sec:related-works} analyzes related works. Section~\ref{sec:method-review} describes the SMS methodology. Section~\ref{sec:threat-validity} addresses threats to validity. Section~\ref{sec:analysis-and-discussion} presents the analysis, discussion, and future research directions. Section~\ref{sec:conclusions} concludes the paper.