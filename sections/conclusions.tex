\section{Conclusions}
\label{sec:conclusions}
This paper presented a Systematic Mapping Study (SMS) on container-based virtualization (CBV) technologies, analyzing 226 primary studies published between 2022 and 2024. By employing a hybrid search strategy and rigorous quality assessment indices (CVI, SCI, IRRQ), the study established a comprehensive overview of the current research landscape. The evidence synthesized leads to the following critical conclusions:

\begin{enumerate}
    \item \textbf{Comprehensive mapping and originality:} This SMS constitutes the first secondary study to simultaneously map CBV technologies across both IT domains and academic dimensions (education, research, and outreach), filling a structural gap identified in all prior literature.
    \item \textbf{Technology monoculture:} A pronounced concentration around Docker (41.6\%) and Kubernetes (29.6\%) was observed. This dominance suggests a research bias that may overlook the industrial and technical advantages of alternative runtimes like Podman or Singularity, and lighter orchestrators such as K3s or Nomad.
    \item \textbf{Domain imbalance:} Research remains heavily anchored to IT Infrastructure (75.66\%), while critical areas such as Blockchain (0.76\%) and Parallel Computing (1.77\%) represent significant "blind spots" that require urgent academic investigation.
    \item \textbf{Academic fragmentation:} Although CBV is inherently suited for environment reproducibility, its adoption is heavily skewed toward Research (83.61\%), with Education (11.06\%) and Outreach (5.88\%) remaining underrepresented. The fact that no study addressed all three dimensions simultaneously indicates that CBV’s transversal potential for holistic academic impact remains largely unrealized.
    \item \textbf{Methodological contribution:} The proposed dual-axis taxonomy provides a novel, structured reference for researchers and practitioners to navigate the rapidly expanding CBV field. Furthermore, the identification of six concrete research gaps (RG1–RG6) offers a clear roadmap for future efforts in benchmarking, security, and pedagogical innovation.
\end{enumerate}

The sustained growth in publication rates—a 118\% increase over the three-year period—confirms the vitality of the field but also warns of potential literature saturation. The taxonomic structure and reproducibility artifacts provided herein aim to mitigate this risk by offering a grounded foundation for the next generation of cloud-native research.