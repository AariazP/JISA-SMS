\section{Conclusions}
\label{sec:conclusions}
This paper presented a Systematic Mapping Study (SMS) on container-based virtualization (CBV) technologies, analyzing 226 primary studies published between 2022 and 2024. By employing a hybrid search strategy and rigorous quality assessment indices (CVI, SCI, IRRQ), the study established a comprehensive overview of the current research landscape. The evidence synthesized leads to the following conclusions:

\begin{enumerate}
    \item \textbf{Comprehensive mapping:} This SMS is, to the best of the authors' knowledge, the first secondary study to simultaneously map CBV technologies across both IT domains and academic dimensions (education, research, and outreach), addressing a gap identified in the reviewed literature.
    \item \textbf{Technology monoculture:} A pronounced concentration around Docker (41.6\%) and Kubernetes (29.6\%) was observed. This concentration suggests that the research literature may not adequately represent the technical characteristics of alternative runtimes like Podman or Singularity, and lighter orchestrators such as K3s or Nomad.
    \item \textbf{Domain imbalance:} Research remains heavily anchored to IT Infrastructure (75.66\%), while critical areas such as Blockchain (0.76\%) and Parallel Computing (1.77\%) represent ``blind spots'' that require further investigation.
    \item \textbf{Academic fragmentation:} Although CBV is inherently suited for environment reproducibility, its adoption is heavily skewed toward Research (83.61\%), with Education (11.06\%) and Outreach (5.88\%) remaining underrepresented. The fact that no study addressed all three dimensions simultaneously indicates that CBV’s transversal potential for holistic academic impact remains largely unrealized.
    \item \textbf{Methodological contribution:} The proposed dual-axis taxonomy provides a structured reference for researchers and practitioners to navigate the rapidly expanding CBV field. Furthermore, the identification of six concrete research gaps (RG1--RG6) provides directions for future efforts in benchmarking, security, and pedagogical development.
\end{enumerate}

The sustained growth in publication rates\textemdash a 118\% increase over the three-year period\textemdash indicates continued interest in the field but also suggests a risk of literature fragmentation. The taxonomic structure and reproducibility artifacts provided herein aim to support future work by offering a structured foundation for the next generation of cloud-native research.