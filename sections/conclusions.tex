\section{Conclusions}
\label{sec:conclusions}
This paper presented a Systematic Mapping Study (SMS) on container-based virtualization (CBV) technologies, covering 226 primary studies published between 2022 and 2024. The study employed a hybrid search strategy (database queries and snowballing), three quality assessment indices (CVI, SCI, IRRQ), and a dual-axis classification spanning 11 IT domains and three academic dimensions (education, research, outreach).

The main contributions and findings are as follows:

\begin{enumerate}
    \item \textbf{Comprehensive mapping:} The SMS provides the first systematic mapping that simultaneously covers CBV technologies across IT domains and academic dimensions, addressing a gap identified in all prior secondary studies.
    \item \textbf{Technology concentration:} Docker (41.6\%) and Kubernetes (29.6\%) dominate the research landscape, while alternative runtimes (Podman, Singularity, gVisor) and orchestrators (K3s, Nomad) remain significantly underrepresented despite their growing industrial relevance.
    \item \textbf{Domain imbalance:} IT Infrastructure accounts for 75.66\% of the corpus, while domains such as Blockchain (0.76\%), Parallel Computing (1.77\%), and Computational Thinking (2.21\%) represent critical blind spots in current research.
    \item \textbf{Academic fragmentation:} Research dominates (83.61\%) while Education (11.06\%) and Outreach (5.88\%) remain underexplored. No study simultaneously addresses all three academic dimensions, indicating that CBV's transversal potential for academic activities remains unrealized.
    \item \textbf{Taxonomic structure:} A novel dual-axis taxonomy organizes the CBV literature by IT domain and academic dimension, providing a structured reference for researchers, educators, and practitioners.
    \item \textbf{Research gaps:} Six concrete gaps (RG1--RG6) were identified, with corresponding future research directions including alternative runtime benchmarking, educational impact studies, outreach frameworks, supply-chain security analysis, and lightweight orchestration for edge/IoT.
\end{enumerate}

The growing publication rate (118\% increase from 2022 to 2024) confirms the vitality of this research area but also underscores the risk of literature saturation without adequate systematization. The taxonomic structure and reproducibility artifacts provided with this study aim to mitigate this challenge.

As future work, we plan to conduct controlled comparative evaluations of container runtimes and orchestrators across heterogeneous hardware platforms, and to design and evaluate container-based pedagogical frameworks for computer science education.
