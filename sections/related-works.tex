%-----------Section 3---------------
\section{Related Works}
\label{sec:related-works}
Several secondary studies have addressed container-based virtualization (CBV) from different perspectives. To position the contribution of the present SMS, this section critically analyzes the existing literature organized by three dimensions: \textit{(i)}~scope and coverage, \textit{(ii)}~methodology, and \textit{(iii)}~domain focus. Table~\ref{tab:related-works-comparison} synthesizes this comparative analysis.

Prior reviews differ substantially in the breadth of technologies and application contexts examined. \cite{Bentaleb2022} and~\cite{SepulvedaRodriguez2022} both propose taxonomic classifications of virtualization technologies; however, neither extends its analysis to the academic dimension (education, research, and outreach). Similarly,~\cite{Malhotra2024} provide a systematic literature review focused exclusively on container lifecycle management---image detection, scheduling, security, and performance---without mapping these concerns to specific IT domains or educational contexts. In contrast,~\cite{Kaiser2022,Kaiser2023} narrow their scope to ARM-compatible container technologies, prioritizing energy efficiency and edge computing performance. While valuable, this architectural focus limits the generalizability of their findings across the full spectrum of CBV use cases.

Among the related works, only~\cite{10094059} adopt a systematic mapping methodology, focusing on container orchestration architectures in cloud computing. Their categorization scheme, however, is restricted to orchestration and does not encompass runtime technologies, academic applications, or cross-domain analysis. The remaining studies employ narrative or traditional review methods, which---while informative---lack the structured, reproducible search protocols and quality assessment mechanisms that characterize an SMS~\cite{Kitchenham2010}.

A common limitation across all reviewed studies is the absence of a cross-cutting analysis that maps CBV technologies to both IT domains (e.g., software development, HPC, security, AI) and academic dimensions simultaneously. None of the existing works:~\textit{(a)}~provides a comprehensive mapping of CBV across multiple IT domains;~\textit{(b)}~examines the role of containerization in education and outreach;~or~\textit{(c)}~offers a taxonomic structure linking technologies, domains, and academic impact.

\begin{table}[htbp]
\centering
\scriptsize
\renewcommand{\arraystretch}{1.1}
\begin{tabularx}{\columnwidth}{>{\raggedright\arraybackslash}p{1.5cm} >{\centering\arraybackslash}p{0.7cm} >{\centering\arraybackslash}p{0.9cm} >{\centering\arraybackslash}p{0.7cm} >{\centering\arraybackslash}p{0.9cm} >{\centering\arraybackslash}p{0.9cm}}
\toprule
\textbf{Study} & \textbf{Type} & \textbf{Multi-domain} & \textbf{Acad.} & \textbf{Taxon.} & \textbf{Reprod.} \\
\midrule
\cite{Bentaleb2022} & Review & \ding{55} & \ding{55} & \ding{51} & \ding{55} \\
\cite{Kaiser2022} & Review & \ding{55} & \ding{55} & \ding{55} & \ding{55} \\
\cite{SepulvedaRodriguez2022} & Review & \ding{55} & \ding{55} & \ding{51} & \ding{55} \\
\cite{Kaiser2023} & Review & \ding{55} & \ding{55} & \ding{55} & \ding{55} \\
\cite{10094059} & SMS & \ding{55} & \ding{55} & \ding{51} & Partial \\
\cite{Malhotra2024} & SLR & \ding{55} & \ding{55} & \ding{55} & Partial \\
\textbf{This study} & \textbf{SMS} & \ding{51} & \ding{51} & \ding{51} & \ding{51} \\
\bottomrule
\end{tabularx}
\caption{Comparative analysis of related secondary studies. Multi-domain: covers multiple IT domains; Acad.: includes academic dimensions; Taxon.: proposes a taxonomy; Reprod.: provides reproducibility artifacts.}\label{tab:related-works-comparison}
\end{table}

This SMS addresses these gaps by providing:~\textit{(1)}~a systematic, reproducible mapping of 226 primary studies across 11 IT domains;~\textit{(2)}~a novel classification linking CBV technologies to education, research, and outreach; and~\textit{(3)}~a taxonomic structure that integrates technological and academic perspectives, enabling researchers and practitioners to identify both consolidated areas and under-explored research opportunities.