% subsection 4.2
\subsection{Stage 2: Study Search}
A hybrid search strategy combining database queries and snowballing was employed. Figure~\ref{fig:etapa2} summarizes the components of this stage.
\begin{figure}[htbp]
    \centering
    \includegraphics[width=0.5\textwidth]{resources/images/planeacion/estrategias-busqueda.png}
    \caption{Composition of the study search stage}
    \label{fig:etapa2}
\end{figure}

%sub-subsection 4.2.1 
\subsubsection{Defining the Search Strategy}\label{subsubsec:Definiendo la Estrategia de Busqueda}
Two complementary strategies were combined. The first involves automated search string execution in academic databases~\cite{jalali2012systematic}. The second, snowballing, identifies additional studies through backward (reference tracking) and forward (citation tracking) analysis of a seed set~\cite{jalali2012systematic,goodman1961snowball}.

%sub-subsection 4.2.2
\subsubsection{Search Strategy 1: Databases}
This strategy comprises two phases: \textit{Study Identification} (search string construction and execution) and \textit{Study Selection} (criteria-based refinement).

\begin{itemize}
    \item \textbf{Study Identification:} Five databases were queried: \textit{ACM}, \textit{IEEE Xplore}, \textit{Springer}, \textit{Science Direct}, and \textit{Taylor \& Francis}. Keywords were derived from the PICOC model (Table~\ref{tab:palabras-clave}) and expanded with synonyms (Table~\ref{tab:keywords}). Boolean operators (\textit{AND}, \textit{OR}) and exact-phrase matching were used to construct database-specific search strings through iterative pilot searches. The complete search strings are available via the reproducibility artifacts (Section~\ref{sec:reproducibility-review-method}).
    
    Execution across all databases yielded \textbf{6,530} preliminary results, with Springer contributing the largest share (\textbf{4,562}; 69.8\%). Table~\ref{tab:resultados-busqueda-sin-criterio} details the distribution.

    \begin{table}[H]
        \caption{Keywords identified using the PICOC model}
        \label{tab:palabras-clave}
        \scriptsize
        \centering
        \setlength{\tabcolsep}{4pt}
        \renewcommand{\arraystretch}{1.05}
        \begin{tabularx}{\columnwidth}{@{}l>{\RaggedRight\arraybackslash}X@{}}
            \toprule
            \textbf{Aspect} & \textbf{Description} \\
            \midrule
            Population & CBV, IT Domains, Education, Research, Outreach \\
            Intervention & Identification, Classification \\
            Comparison & Success rate, Evidence of use \\
            Output & Classification of CBV studies per IT domain \\
            Context & Education, Research, Outreach \\
            \bottomrule
        \end{tabularx}
    \end{table}

\begin{table}[H]
    \caption{Keywords for database search}
    \label{tab:keywords}
    \scriptsize
    \centering
    \setlength{\tabcolsep}{4pt}
    \renewcommand{\arraystretch}{1.05}
    \begin{tabularx}{\columnwidth}{@{}l>{\RaggedRight\arraybackslash}X@{}}
        \toprule
        \textbf{Keyword} & \textbf{Synonyms} \\
        \midrule
        Container-based virtualization & Application virtualization, Docker, Lightweight Virtualization \\
        Education & Education System, Education Development, Higher Education \\
        Research & Research Group, Research Proposal \\
        Industry & IT Services, Technology Infrastructure, Cloud Computing \\
        \bottomrule
    \end{tabularx}
\end{table}

    \item \textbf{Study Selection:} Application of inclusion and exclusion criteria reduced the set to \textbf{976} studies (Table~\ref{tab:resultados-busqueda-criterios}), with Springer maintaining the largest contribution (\textbf{592}; 60.65\%). After removing \textbf{274} duplicates, a screening process (title, abstract, and keyword review) excluded \textbf{593} irrelevant studies, yielding \textbf{110} selected studies from the database strategy. Figure~\ref{fig:resumen-busqueda-bds} summarizes this process.
\end{itemize}
\begin{figure}[htbp]
    \centering
    \includegraphics[width=0.5\textwidth]{resources/images/busqueda-estudios/busqueda-bd.png}
    \caption{Summary of activities and results obtained in the database search strategy}
    \label{fig:resumen-busqueda-bds}
\end{figure}

\begin{table*}[tbp]
    \caption{Search results per database using keywords}\label{tab:resultados-busqueda-sin-criterio}
    \scriptsize
    \centering
    \setlength{\tabcolsep}{4pt}
    \renewcommand{\arraystretch}{1.05}
    \begin{tabular}{@{}lrrrrr@{}}
        \toprule
        \textbf{Criterion} & \textbf{ACM} & \textbf{IEEE} & \textbf{Science Direct} & \textbf{Springer} & \textbf{Taylor and Francis} \\
        \midrule
        Search results using keywords only & 189 & 426 & 4562 & 353 & 1000 \\
        Contribution percentage & 2.89\% & 6.52\% & 69.86\% & 5.4\% & 15.31\% \\
        \bottomrule
    \end{tabular}
\end{table*}

\begin{table*}[tbp]
    \caption{Search results per database using keywords after applying inclusion/exclusion criteria}\label{tab:resultados-busqueda-criterios}
    \scriptsize
    \centering
    \setlength{\tabcolsep}{4pt}
    \renewcommand{\arraystretch}{1.05}
    \begin{tabular}{@{}lrrrrr@{}}
        \toprule
        \textbf{Criterion} & \textbf{ACM} & \textbf{IEEE} & \textbf{Science Direct} & \textbf{Springer} & \textbf{Taylor and Francis} \\
        \midrule
        Search results after applying keywords only & 48 & 134 & 46 & 592 & 156 \\
        Contribution percentage & 4.91\% & 13.72\% & 4.71\% & 60.65\% & 15.98\% \\
        \bottomrule
    \end{tabular}
\end{table*}

% sub-subsection 4.2.3
\subsubsection{Search Strategy 2: Snowballing}
The snowballing search strategy began with the identification of the base set of articles. This base set was obtained from Search Strategy 1. The procedure consisted of two phases:

The first phase, called \textit{Baseline Construction}, aimed to establish the articles on which a citation and reference analysis would be performed. To form this initial set of studies, several criteria were applied, including the CVI \textit{(Content Validity Index)}, the SCI \textit{(Scientific Citation Index)}, and the direct inclusion criterion. The second phase, called \textit{Study Selection}, focused on the analysis of references \textit{(Backward Snowballing)} and citations \textit{(Forward Snowballing)} corresponding to each article \cite{10.1145/2601248.2601268}.

Baseline construction started from the \textbf{110} articles obtained through the database search strategy. From this set, \textbf{25} articles were selected using the SCI quality criterion. This criterion was chosen because it relies on the number of citations received by each article rather than on evaluator judgment, thus providing an objective indicator of academic impact. Selection was performed through citation frequency analysis, extracting the first quartile \textbf{(Q1)} corresponding to the most-cited articles.
As part of the SMS process, studies can also be incorporated through direct inclusion, whereby an article previously known to the authors is added without originating from a database search. This procedure adds flexibility by allowing the integration of works deemed relevant to the research objectives. In this case, one article was incorporated through direct inclusion, bringing the total to \textbf{26} articles in the baseline.

After baseline construction, reference analysis was performed. The forward search was conducted using Google Scholar following the practices described in~\cite{8747000}, identifying a total of \textbf{495} new articles. The backward search yielded \textbf{87} additional articles.

\textbf{14} duplicate articles were removed from the backward and forward search results. Subsequently, the \textit{Screening} process was applied again, consisting of title, abstract, and keyword review as in the previous phase. This procedure reduced the set to \textbf{116} articles selected through the snowballing search strategy. Figure~\ref{fig:resumen-busqueda-snowballing} presents a summary of the process followed in this search strategy.

% sub-subsection 4.2.4
\subsubsection{Results of the Study Search}\label{subsubsec:resultados-busqueda}
The combined search yielded \textbf{226} primary studies: \textbf{110} from databases, \textbf{115} from snowballing, and \textbf{1} through direct inclusion. The nearly-equal split between strategies (Table~\ref{tab:resultados-busqueda}) confirms the complementarity of the hybrid approach.
\begin{table}[H]
\caption{Results of the study search}\label{tab:resultados-busqueda}
\centering
\scriptsize
\setlength{\tabcolsep}{4pt}
\renewcommand{\arraystretch}{1.05}
\begin{tabular}{@{}lrr@{}}
\toprule
\textbf{Strategy} & \textbf{Studies} & \textbf{\%} \\
\midrule
Databases & 110 & 48.67\% \\
Snowballing & 115 & 50.88\% \\
Direct Inclusion & 1 & 0.44\% \\
\textbf{Total} & \textbf{226} & \textbf{100\%} \\
\bottomrule
\end{tabular}
\end{table}
\begin{figure}[htbp]
    \centering
    \includegraphics[width=0.5\textwidth]{resources/images/busqueda-estudios/busqueda-snowball.png}
    \caption{Summary of the snowballing search strategy}
    \label{fig:resumen-busqueda-snowballing}
\end{figure}