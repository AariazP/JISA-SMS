% subseccion 4.3
\subsection{Stage 3: Quality Assessment}
Although quality assessment is not mandatory in an SMS~\cite{8747000}, incorporating it strengthens the rigor of the mapping and brings it closer to a systematic review~\cite{10.1145/2601248.2601268}. Three complementary indices---CVI, SCI, and IRRQ---were applied to evaluate study relevance.

%sub-subseccion 4.3.1
\subsubsection{Content Validity Assessment (CVI)}
Each study was independently rated by an odd number of evaluators ($K \geq 3$) on a 0--5 relevance scale. The proportion-based I-CVI (Equation~\ref{eq:cvi}) was computed for each study, and studies with I-CVI $\geq 0.78$ were considered to have acceptable content validity. Two assessment rounds were conducted: the first during baseline construction for snowballing (Section~\ref{subsubsec:resultados-busqueda}), and the second after all 226 studies were identified, with results reported in Section~\hyperref[sec:clasificacion-estudios]{Study Classification}.

%sub-subseccion 4.3.2
\subsubsection{Citation-Based Quality Assessment (SCI)}
The SCI (Equation~\ref{eq:sci}) was computed using citation data from Google Scholar and the SMS-Builder tool~\cite{candela2020smsbuilder}. A frequency analysis identified the top quartile (Q1) of studies by SCI, representing those with the highest citation-normalized impact.

%sub-subseccion 4.3.3
\subsubsection{Research Question Coverage Assessment (IRRQ)}
The IRRQ (Equation~\ref{eq:irrq}) was computed for each study based on its thematic alignment with Q1 and Q2, as determined through the classification process. Studies with IRRQ~$= 1$ (addressing both research questions) were identified through frequency analysis as the most thematically comprehensive.