%-----------Section 4.1---------------
\subsection{Planning}\label{subsec:planeacion}
The planning stage established the research goals, questions, metrics, classification criteria, and quality assessment indices. Figure~\ref{fig:etapa1} summarizes the components of this stage.

\begin{figure}[htbp]
    \centering
    \includegraphics[width=0.5\textwidth]{resources/images/planeacion/etapa1.png}
    \caption{Composition of the planning stage}\label{fig:etapa1}
\end{figure}

%-----------Section 4.1.1---------------
\subsubsection{Study Goals}
Two overarching goals were defined to guide this SMS, as presented in Table~\ref{tab:metas-del-estudio}.

\begin{table}[htbp]
    \centering
    \begin{tabular}{>{\centering\arraybackslash}m{1cm} >{\arraybackslash}m{7cm}}
        \hline
        \textbf{Goal} & \textbf{Description} \\
        \hline\\
        G1 & Identify studies related to CBV in education, research, and outreach. \\
        \\
        G2 & Classify studies related to CBV across IT domains, including software development, computational thinking, parallel computing, data analysis, artificial intelligence, computer networks, IT infrastructure, HPC, security, cloud computing, and blockchain. \\\\
        
        \hline
    \end{tabular}
    \caption{Goals of the study}\label{tab:metas-del-estudio}
\end{table}

\begin{table*}[!t]
\centering

\renewcommand{\arraystretch}{1.4}
\begin{tabularx}{\textwidth}{>{\centering\arraybackslash}m{0.05\textwidth} >{\centering\arraybackslash}m{0.08\textwidth} >{\RaggedRight\arraybackslash}X >{\RaggedRight\arraybackslash}X}
\toprule
\textbf{Goal} & \textbf{Question} & \textbf{Description} & \textbf{Motivation} \\
\midrule
G1 & Q1 & Which studies related to container-based virtualization (CBV) technologies contribute to education, research, and outreach? & CBV enables environment reproducibility, facilitating the transfer of IT solutions across contexts. Understanding its academic penetration can stimulate cross-domain innovation. \\
\midrule
G2 & Q2 & Which primary studies related to CBV technologies contribute to IT domains such as software development, HPC, AI, security, cloud computing, and others? & The goal is to provide a structured overview of CBV adoption across IT domains, enabling researchers and practitioners to identify trends without requiring exhaustive primary analysis. \\
\bottomrule
\end{tabularx}
\caption{Research questions and their motivation}\label{tab:descripcion-preguntas}
\end{table*}
%------Section 4.1.2-----------------------
\subsubsection{Research Questions}
\label{sec:research-question}
The research questions were formulated using the GQM (\textit{Goal Question Metric}) framework~\cite{Needleman2002} and the PICOC model~\cite{petticrew2008systematic}, which structures the population, intervention, comparison, outcome, and context of the study (Table~\ref{tab:aspectos-PICOC}). Two research questions were derived, as detailed in Table~\ref{tab:descripcion-preguntas}.

%------Section 4.1.3-----------------------
\subsubsection{Metrics}
\label{sec:metrics}
Quantitative metrics were defined to measure the distribution of studies across the classification structure (Table~\ref{tab:metricas}). The search period was restricted to 2022--2024 to ensure currency.

\begin{table}[H]
    \scriptsize % reduce text size
    \centering
    \renewcommand{\arraystretch}{1.3}
    \begin{tabularx}{\columnwidth}{>{\centering\arraybackslash}m{0.18\columnwidth} >{\RaggedRight\arraybackslash}X}
        \hline
        \textbf{Aspect} & \textbf{Description} \\
        \hline
        Population & Studies related to CBV applied across IT domains, with emphasis on education, research, and outreach. \\
        Intervention & Identification and classification of CBV studies within established IT domains. \\
        Comparison & 
        1. Comparison of CBV projects by reported success rates across IT domains. \newline
        2. Analysis of CBV impact on academic activities relative to alternative solutions. \\
        Outcome & Classification structure mapping CBV studies to IT domains and academic dimensions. \\
        Context & Education, research, and outreach contexts adopting CBV technologies. \\
        \hline
    \end{tabularx}
    \caption{PICOC model specification}\label{tab:aspectos-PICOC}
\end{table}

%------Section 4.1.4-----------------------
\subsubsection{Research Topics}
\label{sec:research-topics}
Based on the research questions and PICOC model, four research topics were defined: \textit{Container-based virtualization}, \textit{Education}, \textit{Research}, and \textit{Industry}. These topics were further refined through the IT domains identified as relevant to the study scope.

\subsubsection{Inclusion and Exclusion Criteria}
\begin{table*}[!t]
\centering
\renewcommand{\arraystretch}{1.4}
\begin{tabularx}{\textwidth}{>{\centering\arraybackslash}m{0.15\textwidth} >{\RaggedRight\arraybackslash}X >{\RaggedRight\arraybackslash}X}
\toprule
\textbf{Category} & \textbf{Inclusion} & \textbf{Exclusion} \\
\midrule
Screening field & Abstract & --- \\
\midrule
Publication type & Journal articles and conference proceedings & Theses, book chapters, grey literature \\
\midrule
Discipline & Computer Science, Information Technology, Engineering, IT Management & Disciplines unrelated to virtualization or computing \\
\midrule
Time period & 2022--2024 & Before 2022 \\
\midrule
Language & English & Non-English publications \\
\bottomrule
\end{tabularx}
\caption{Inclusion and exclusion criteria}\label{tab:criterios}
\end{table*}

\label{sec:inclusion-exclusion}
Table~\ref{tab:criterios} presents the inclusion and exclusion criteria. The three-year window (2022--2024) balances currency with sufficient volume. Both journal articles and conference proceedings were included to capture the full publication landscape in this rapidly evolving field.\\

\begin{table}[H]
\centering
\renewcommand{\arraystretch}{1.3}
\begin{tabularx}{\columnwidth}{>{\centering\arraybackslash}m{0.15\textwidth} >{\RaggedRight\arraybackslash}X}
\toprule
\textbf{Metric} & \textbf{Description} \\
\midrule
M1 & Number of studies identified per IT domain. \\
M2 & Number of studies classified under education. \\
M3 & Number of studies classified under research. \\
M4 & Number of studies classified under outreach. \\
\bottomrule
\end{tabularx}
\caption{Metrics defined for the analysis}\label{tab:metricas}
\end{table}

%------Section 4.1.6-----------------------
\subsubsection{Quality Assessment Criteria}
Three quality indices were defined to evaluate the relevance and rigor of the selected studies.

\paragraph{Content Validity Index (CVI).}
The CVI assesses the degree to which each study aligns with the SMS objectives, adapted from established content validity methodology~\cite{almanasreh2019evaluation,yaghmaei2003content}. Each study was independently rated by $K$~evaluators (where $K$ is odd, to prevent ties) on a scale from 0~(no relevance) to~5 (high relevance). Following the proportion-based CVI approach~\cite{almanasreh2019evaluation}, we define the item-level CVI (I-CVI) as the proportion of evaluators who rate a study above a relevance threshold $t$:
\begin{equation}
\label{eq:cvi}
\text{I-CVI} = \frac{n_{t}}{K}
\end{equation}
where $n_{t}$ is the number of evaluators assigning a score $\geq t$ (with $t = 3$ adopted in this study), and $K$ is the total number of evaluators. An I-CVI $\geq 0.78$ indicates acceptable content validity~\cite{almanasreh2019evaluation}. For aggregation across the study corpus, the Scale-level CVI based on the average method (S-CVI/Ave) is computed as the mean of all I-CVI values.

\paragraph{Scientific Citation Index (SCI).}
The SCI captures citation-normalized impact relative to publication recency. For a study with $C$~citations accumulated between 2022 and 2024, published $A$~years before the extraction date:
\begin{equation}
\label{eq:sci}
\text{SCI} = \frac{C}{A}
\end{equation}
This normalization ensures that recently published studies with emerging citation counts are not penalized relative to older, more-cited works.

\paragraph{Index of Relationship to Research Questions (IRRQ).}
The IRRQ quantifies the coverage of a study with respect to the defined research questions. Given $Q = 2$ research questions in this SMS, the IRRQ for a study addressing $n$ questions is:
\begin{equation}
\label{eq:irrq}
\text{IRRQ} = \frac{n}{Q}
\end{equation}
where $n \in \{0, 1, 2\}$ and $Q = 2$. Thus, IRRQ $\in \{0, 0.5, 1\}$, where 1~indicates full coverage of both research questions. This index enables identification of studies with broad versus narrow thematic relevance.