% subseccion 4.6
\subsection{Stage 6: Results}
This section presents and interprets the findings from the SMS, organized in three parts: (1)~an overview of the SPS corpus with source and temporal analysis, (2)~a technology and domain distribution analysis with interpretation of observed trends, and (3)~a keyword co-occurrence analysis.

%sub-subseccion 4.6.1
\subsubsection{Overview of the SPS Corpus}
The SMS identified \textbf{226} selected primary studies (SPS), listed in Table~\ref{tab:sps-list}. The following subsections analyze the distribution of these studies across sources, technologies, academic dimensions, and quality indices.

\paragraph{Source and strategy distribution.}
Table~\ref{tab:clasificacion-topicos} presents the classification of CBV technologies by academic dimension. Docker dominates across education, research, and outreach, accounting for \textbf{100} SPS (44.24\%). This predominance reflects Docker's mature ecosystem, extensive documentation, and low barrier to entry, which collectively facilitate adoption in academic contexts where ease of deployment is prioritized over specialized performance characteristics.\\
Table~\ref{tab:clasificacion-sps} maps IT domains to academic dimensions. IT Infrastructure is the most represented domain, with \textbf{171} SPS (75.66\%), indicating that containerization research remains strongly anchored to infrastructure-level concerns such as deployment, scaling, and resource management.
\onecolumn
\begin{longtable*}{c>{\centering\arraybackslash}p{6cm}
                 c>{\centering\arraybackslash}p{6cm}}
\label{tab:sps-list} \\
\textbf{ID} & \textbf{Ref} &
\textbf{ID} & \textbf{Ref} \\
\hline
\endfirsthead

\multicolumn{4}{c}{{\bfseries \tablename\ \thetable{} -- continued}} \\
\textbf{ID} & \textbf{Ref} &
\textbf{ID} & \textbf{Ref} \\
\hline
\endhead

\hline \multicolumn{4}{r}{{Continues on the next page}} \\
\endfoot

\hline
\endlastfoot
SPS001 & \citeauthor{1} & SPS002 & \citeauthor{2} \\
SPS003 & \citeauthor{3} & SPS004 & \citeauthor{4} \\
SPS005 & \citeauthor{5} & SPS006 & \citeauthor{6} \\
SPS007 & \citeauthor{7} & SPS008 & \citeauthor{8} \\
SPS009 & \citeauthor{9} & SPS010 & \citeauthor{10} \\
SPS011 & \citeauthor{11} & SPS012 & \citeauthor{12} \\
SPS013 & \citeauthor{13} & SPS014 & \citeauthor{14} \\
SPS015 & \citeauthor{15} & SPS016 & \citeauthor{16} \\
SPS017 & \citeauthor{17} & SPS018 & \citeauthor{18} \\
SPS019 & \citeauthor{19} & SPS020 & \citeauthor{20} \\
SPS021 & \citeauthor{21} & SPS022 & \citeauthor{22} \\
SPS023 & \citeauthor{23} & SPS024 & \citeauthor{24} \\
SPS025 & \citeauthor{25} & SPS026 & \citeauthor{26} \\
SPS027 & \citeauthor{27} & SPS028 & \citeauthor{28} \\
SPS029 & \citeauthor{29} & SPS030 & \citeauthor{30} \\
SPS031 & \citeauthor{31} & SPS032 & \citeauthor{32} \\
SPS033 & \citeauthor{33} & SPS034 & \citeauthor{34} \\
SPS035 & \citeauthor{35} & SPS036 & \citeauthor{36} \\
SPS037 & \citeauthor{37} & SPS038 & \citeauthor{38} \\
SPS039 & \citeauthor{39} & SPS040 & \citeauthor{40} \\
SPS041 & \citeauthor{41} & SPS042 & \citeauthor{42} \\
SPS043 & \citeauthor{43} & SPS044 & \citeauthor{44} \\
SPS045 & \citeauthor{45} & SPS046 & \citeauthor{46} \\
SPS047 & \citeauthor{47} & SPS048 & \citeauthor{48} \\
SPS049 & \citeauthor{49} & SPS050 & \citeauthor{50} \\
SPS051 & \citeauthor{51} & SPS052 & \citeauthor{52} \\
SPS053 & \citeauthor{53} & SPS054 & \citeauthor{54} \\
SPS055 & \citeauthor{55} & SPS056 & \citeauthor{56} \\
SPS057 & \citeauthor{57} & SPS058 & \citeauthor{58} \\
SPS059 & \citeauthor{59} & SPS060 & \citeauthor{60} \\
SPS061 & \citeauthor{61} & SPS062 & \citeauthor{62} \\
SPS063 & \citeauthor{63} & SPS064 & \citeauthor{64} \\
SPS065 & \citeauthor{65} & SPS066 & \citeauthor{66} \\
SPS067 & \citeauthor{67} & SPS068 & \citeauthor{68} \\
SPS069 & \citeauthor{69} & SPS070 & \citeauthor{70} \\
SPS071 & \citeauthor{71} & SPS072 & \citeauthor{72} \\
SPS073 & \citeauthor{73} & SPS074 & \citeauthor{74} \\
SPS075 & \citeauthor{75} & SPS076 & \citeauthor{76} \\
SPS077 & \citeauthor{77} & SPS078 & \citeauthor{78} \\
SPS079 & \citeauthor{79} & SPS080 & \citeauthor{80} \\
SPS081 & \citeauthor{81} & SPS082 & \citeauthor{82} \\
SPS083 & \citeauthor{83} & SPS084 & \citeauthor{84} \\
SPS085 & \citeauthor{85} & SPS086 & \citeauthor{86} \\
SPS087 & \citeauthor{87} & SPS088 & \citeauthor{88} \\
SPS089 & \citeauthor{89} & SPS090 & \citeauthor{90} \\
SPS091 & \citeauthor{91} & SPS092 & \citeauthor{92} \\
SPS093 & \citeauthor{93} & SPS094 & \citeauthor{94} \\
SPS095 & \citeauthor{95} & SPS096 & \citeauthor{96} \\
SPS097 & \citeauthor{97} & SPS098 & \citeauthor{98} \\
SPS099 & \citeauthor{99} & SPS100 & \citeauthor{100} \\
SPS101 & \citeauthor{101} & SPS102 & \citeauthor{102} \\
SPS103 & \citeauthor{103} & SPS104 & \citeauthor{104} \\
SPS105 & \citeauthor{105} & SPS106 & \citeauthor{106} \\
SPS107 & \citeauthor{107} & SPS108 & \citeauthor{108} \\
SPS109 & \citeauthor{109} & SPS110 & \citeauthor{110} \\
SPS111 & \citeauthor{111} & SPS112 & \citeauthor{112} \\
SPS113 & \citeauthor{113} & SPS114 & \citeauthor{114} \\
SPS115 & \citeauthor{115} & SPS116 & \citeauthor{116} \\
SPS117 & \citeauthor{117} & SPS118 & \citeauthor{118} \\
SPS119 & \citeauthor{119} & SPS120 & \citeauthor{120} \\
SPS121 & \citeauthor{121} & SPS122 & \citeauthor{122} \\
SPS123 & \citeauthor{123} & SPS124 & \citeauthor{124} \\
SPS125 & \citeauthor{125} & SPS126 & \citeauthor{126} \\
SPS127 & \citeauthor{127} & SPS128 & \citeauthor{128} \\
SPS129 & \citeauthor{129} & SPS130 & \citeauthor{130} \\
SPS131 & \citeauthor{131} & SPS132 & \citeauthor{132} \\
SPS133 & \citeauthor{133} & SPS134 & \citeauthor{134} \\
SPS135 & \citeauthor{135} & SPS136 & \citeauthor{136} \\
SPS137 & \citeauthor{137} & SPS138 & \citeauthor{138} \\
SPS139 & \citeauthor{139} & SPS140 & \citeauthor{140} \\
SPS141 & \citeauthor{141} & SPS142 & \citeauthor{142} \\
SPS143 & \citeauthor{143} & SPS144 & \citeauthor{144} \\
SPS145 & \citeauthor{145} & SPS146 & \citeauthor{146} \\
SPS147 & \citeauthor{147} & SPS148 & \citeauthor{148} \\
SPS149 & \citeauthor{149} & SPS150 & \citeauthor{150} \\
SPS151 & \citeauthor{151} & SPS152 & \citeauthor{152} \\
SPS153 & \citeauthor{153} & SPS154 & \citeauthor{154} \\
SPS155 & \citeauthor{155} & SPS156 & \citeauthor{156} \\
SPS157 & \citeauthor{157} & SPS158 & \citeauthor{158} \\
SPS159 & \citeauthor{159} & SPS160 & \citeauthor{160} \\
SPS161 & \citeauthor{161} & SPS162 & \citeauthor{162} \\
SPS163 & \citeauthor{163} & SPS164 & \citeauthor{164} \\
SPS165 & \citeauthor{165} & SPS166 & \citeauthor{166} \\
SPS167 & \citeauthor{167} & SPS168 & \citeauthor{168} \\
SPS169 & \citeauthor{169} & SPS170 & \citeauthor{170} \\
SPS171 & \citeauthor{171} & SPS172 & \citeauthor{172} \\
SPS173 & \citeauthor{173} & SPS174 & \citeauthor{174} \\
SPS175 & \citeauthor{175} & SPS176 & \citeauthor{176} \\
SPS177 & \citeauthor{177} & SPS178 & \citeauthor{178} \\
SPS179 & \citeauthor{179} & SPS180 & \citeauthor{180} \\
SPS181 & \citeauthor{181} & SPS182 & \citeauthor{182} \\
SPS183 & \citeauthor{183} & SPS184 & \citeauthor{184} \\
SPS185 & \citeauthor{185} & SPS186 & \citeauthor{186} \\
SPS187 & \citeauthor{187} & SPS188 & \citeauthor{188} \\
SPS189 & \citeauthor{189} & SPS190 & \citeauthor{190} \\
SPS191 & \citeauthor{191} & SPS192 & \citeauthor{192} \\
SPS193 & \citeauthor{193} & SPS194 & \citeauthor{194} \\
SPS195 & \citeauthor{195} & SPS196 & \citeauthor{196} \\
SPS197 & \citeauthor{197} & SPS198 & \citeauthor{198} \\
SPS199 & \citeauthor{199} & SPS200 & \citeauthor{200} \\
SPS201 & \citeauthor{201} & SPS202 & \citeauthor{202} \\
SPS203 & \citeauthor{203} & SPS204 & \citeauthor{204} \\
SPS205 & \citeauthor{205} & SPS206 & \citeauthor{206} \\
SPS207 & \citeauthor{207} & SPS208 & \citeauthor{208} \\
SPS209 & \citeauthor{209} & SPS210 & \citeauthor{210} \\
SPS211 & \citeauthor{211} & SPS212 & \citeauthor{212} \\
SPS213 & \citeauthor{213} & SPS214 & \citeauthor{214} \\
SPS215 & \citeauthor{215} & SPS216 & \citeauthor{216} \\
SPS217 & \citeauthor{217} & SPS218 & \citeauthor{218} \\
SPS219 & \citeauthor{219} & SPS220 & \citeauthor{220} \\
SPS221 & \citeauthor{221} & SPS222 & \citeauthor{222} \\
SPS223 & \citeauthor{223} & SPS224 & \citeauthor{224} \\
SPS225 & \citeauthor{225} & SPS226 & \citeauthor{226} \\
\end{longtable*}
\twocolumn
\paragraph{Technology landscape analysis.}
Figure~\ref{fig:SPS-SOURCE-SE} shows the distribution of studies by source and strategy. Of the database-sourced studies (110~SPS), IEEE Xplore and ACM Digital Library jointly contribute 68.18\%, reflecting the strong alignment of CBV research with computing-focused venues. The snowballing strategy (115~SPS) was dominated by forward snowballing (92.17\%), suggesting that CBV is an expanding field where newer publications actively cite foundational works.
\begin{figure}[htpb]
    \centering
    \includegraphics[width=0.5\textwidth]{resources/images/resultados/SPS-SOURCE-SE.png}
    \caption{SPS by source and search strategy}\label{fig:SPS-SOURCE-SE}
\end{figure}

Figure~\ref{fig:SPS-VBC} reveals the distribution of container runtime technologies. Docker leads with \textbf{94} SPS, followed distantly by Podman (7), LXC and Containerd (4 each), and Singularity, runC, and gVisor (3 each). This concentration raises an important finding: \textit{despite the growing ecosystem of alternative container runtimes designed for security (gVisor, Kata Containers), HPC (Singularity), and rootless operation (Podman), the research community remains heavily Docker-centric}. This gap between technological diversity and research coverage represents an opportunity for future studies to evaluate emerging runtimes in domains where Docker's limitations are well documented.

\begin{center}
    \includegraphics[width=0.5\textwidth]{resources/images/resultados/SPS-VBC.png}
    \captionof{figure}{Distribution of container runtime technologies across SPS}
    \label{fig:SPS-VBC}
\end{center}
Figure~\ref{fig:SPS-ORCH} shows orchestrator distribution. Kubernetes dominates with \textbf{67} SPS, confirming its status as the \textit{de facto} standard for container orchestration. Docker Swarm (9~SPS) and Apache Mesos (5~SPS) trail significantly. The marginal representation of alternatives such as OpenShift (2), Docker Compose (3), and cloud-native services (Amazon ECS/EKS, 1 each) suggests that \textit{academic research has not yet systematically evaluated the trade-offs between Kubernetes and its alternatives}, particularly in edge computing, serverless, and resource-constrained environments where lighter orchestration solutions may be more appropriate.
\begin{center}
    \includegraphics[width=0.5\textwidth]{resources/images/resultados/orch-SPS.png}
    \captionof{figure}{Distribution of orchestrator technologies across SPS}
    \label{fig:SPS-ORCH}
\end{center}

\paragraph{Academic dimension analysis.}
Figure~\ref{fig:SPS-venn} illustrates the intersection of studies across academic dimensions. Research dominates with \textbf{187} exclusive SPS, while Education (19~exclusive) and Outreach (8~exclusive) remain underrepresented. Notably, \textit{no study simultaneously addresses all three dimensions}, revealing a significant fragmentation in academic production. Only 6~SPS bridge Research and Education, and 6~bridge Research and Outreach, with zero overlap between Education and Outreach. This finding suggests that the potential of CBV as a transversal tool linking teaching, research output, and societal impact remains largely unexplored.
\begin{center}
    \includegraphics[width=0.5\textwidth]{resources/images/resultados/SPS-venn.png}
    \captionof{figure}{Venn diagram of SPS across academic dimensions}
    \label{fig:SPS-venn}
\end{center}

\paragraph{Topic and temporal distribution.}
Figure~\ref{fig:SPS-topics} shows topic distribution per research question. For Q1, Research accounts for 83.61\% of studies, while Outreach represents only 5.88\%---a disparity that underscores the limited penetration of containerization into community engagement and societal applications. For Q2, IT Infrastructure leads (41.28\%), followed by Cloud Computing (14.37\%), while Blockchain (0.76\%) and Parallel Computing represent emerging but underexplored intersections with CBV.
\begin{figure}[htbp]
    \centering
    \includegraphics[width=0.5\textwidth]{resources/images/resultados/SPS-topics.png}
    \caption{SPS distribution by research questions and topics}\label{fig:SPS-topics}
\end{figure}

The temporal analysis (Figure~\ref{fig:SPS-QI}) reveals sustained growth, from \textbf{49} SPS in 2022 to \textbf{107} in 2024 (a 118\% cumulative increase). The sharpest growth occurred between 2023 and 2024 (+52.85\%), coinciding with the maturation of Kubernetes-based cloud-native architectures and the proliferation of edge computing applications. The CVI index shows an upward trend (from 7 to 9, +28.57\%), suggesting that more recent studies exhibit stronger alignment with the SMS objectives. The SCI index remains stable around 18, with a slight recovery in 2024 (+11.76\% over 2023), while the IRRQ index exhibits sustained growth from 30 to 39 (+30\% cumulative), indicating increasing thematic breadth in newer publications.

\begin{figure}[htbp]
    \centering
    \includegraphics[width=0.5\textwidth]{resources/images/resultados/SPS-QI.png}
    \caption{SPS by year and quality indices}\label{fig:SPS-QI}
\end{figure}

Figure~\ref{fig:SPS-topics-QI} presents the quality indices disaggregated by topic. IT Infrastructure not only has the highest volume (73~SPS, 43.71\% of Q2) but also concentrates the highest-quality studies across all three indices, reinforcing its centrality to the CBV research landscape. The Outreach topic, with only 8~SPS (7.33\% of Q1), represents the most significant gap identified in this mapping.

Figure~\ref{fig:SPS-kw} presents a cross-analysis of keywords. The term ``\textit{Container}'' enables the identification of 57~SPS, while education-related keywords (\textit{Learning}, \textit{Cybersecurity education}) appear in only 2~SPS each---further evidence that the academic community has not yet developed a robust vocabulary linking CBV to educational and outreach applications.

\onecolumn

\input{figures/study-clasification-by-topic-and-year}
\vspace{1cm}
\begin{center}
\renewcommand{\arraystretch}{1.15}
\begin{longtable}{c Z Y Y Y}
\caption{Studies with the highest CVI index, classified by topics}\label{tab:higher-cvi} \\

\toprule
\textbf{RQ} & \textbf{Topics} & \textbf{2022} & \textbf{2023} & \textbf{2024} \\
\midrule
\endfirsthead

\toprule
\textbf{RQ} & \textbf{Topics} & \textbf{2022} & \textbf{2023} & \textbf{2024} \\
\midrule
\endhead

\multirow{3}{*}{\textbf{Q1}} 
& Research & SPS003, SPS007, SPS083, SPS145, SPS146 & SPS068, SPS174 & SPS032, SPS136, SPS151, SPS168 \\
& Education & SPS038, SPS146 & SPS152, SPS206 & SPS089, SPS115, SPS151 \\
\midrule
\multirow{10}{*}{\textbf{Q2}} 
& Software Development & SPS038 & & \\
& Computational Thinking & & & SPS115 \\
& Data Analysis & SPS037, SPS071, SPS157 & SPS183, SPS209 & SPS001, SPS005, SPS028, SPS045, SPS061, SPS082, SPS129 \\
& IT Infrastructure & SPS003, SPS007, SPS038, SPS083, SPS145, SPS146 & SPS068, SPS152, SPS174, SPS206 & SPS032, SPS089, SPS115, SPS136, SPS151, SPS168 \\
& HPC & SPS083 & & \\
& Security & SPS083 & & \\
& Cloud Computing & SPS003, SPS146 & & SPS032, SPS136 \\
\bottomrule
\end{longtable}
\end{center}

\vspace{10cm}
\begin{center}
\renewcommand{\arraystretch}{1.15}
\begin{longtable}{c Z Y Y Y}
\caption{Studies with the Highest SCI Index, Categorized by Topic}\label{tab:higher-sci} \\

\toprule
\textbf{RQ} & \textbf{Topics} & \textbf{2022} & \textbf{2023} & \textbf{2024} \\
\midrule
\endfirsthead

\toprule
\textbf{RQ} & \textbf{Topics} & \textbf{2022} & \textbf{2023} & \textbf{2024} \\
\midrule
\endhead

\multirow{3}{*}{\textbf{Q1}} 
& Research & SPS003, SPS044, SPS064, SPS083, SPS092, SPS137, SPS143, SPS145, SPS157, SPS176, SPS187, SPS192 & SPS027, SPS029, SPS126, SPS165, SPS173, SPS223 & SPS028, SPS032, SPS033, SPS054, SPS140, SPS197, SPS215 \\
& Education & SPS187 & SPS020, SPS072 & \\
\midrule
\multirow{10}{*}{\textbf{Q2}} 
& Software development & SPS044 & & SPS028, SPS215 \\
& Computational thinking & SPS187 & & \\
& Parallel computing & & SPS020, SPS223 & \\
& Data analysis & SPS157 & & SPS028 \\
& Artificial intelligence & & SPS027, SPS072 & \\
& Computer networks & SPS187 & & \\
& IT infrastructure & SPS003, SPS083, SPS092, SPS137, SPS143, SPS145, SPS176, SPS187 & SPS020, SPS027, SPS029, SPS126, SPS173, SPS223 & SPS032, SPS033, SPS054, SPS140, SPS197, SPS215 \\
& HPC & SPS083 & SPS027 & \\
& Security & SPS064, SPS083, SPS092, SPS157, SPS192 & SPS126, SPS165 & \\
& Cloud computing & SPS003, SPS137, SPS143 & SPS029, SPS126, SPS173 & SPS032, SPS033, SPS197 \\
\bottomrule
\end{longtable}
\end{center}
\vspace{10cm}
\begin{center}
\renewcommand{\arraystretch}{1.15}
\begin{longtable}{c Z Y Y Y}
\caption{Studies with the Highest IRRQ Index, Classified by Topic}\label{tab:higher-irrq} \\

\toprule
\textbf{RQ} & \textbf{Topics} & \textbf{2022} & \textbf{2023} & \textbf{2024} \\
\midrule
\endfirsthead

\toprule
\textbf{RQ} & \textbf{Topics} & \textbf{2022} & \textbf{2023} & \textbf{2024} \\
\midrule
\endhead

\multirow{3}{*}{\textbf{Q1}}
& Research & SPS002, SPS003, SPS007, SPS039, SPS044, SPS053, SPS059, SPS064, SPS070, SPS071, SPS073, SPS080, SPS083, SPS092, SPS137, SPS143, SPS145, SPS146, SPS155, SPS157, SPS176, SPS177, SPS187, SPS192 & SPS027, SPS029, SPS055, SPS066, SPS067, SPS068, SPS081, SPS093, SPS094, SPS117, SPS126, SPS134, SPS153, SPS165, SPS167, SPS173, SPS174, SPS183, SPS195, SPS205, SPS209, SPS221, SPS223, SPS226 & SPS005, SPS008, SPS010, SPS019, SPS021, SPS028, SPS030, SPS032, SPS033, SPS036, SPS045, SPS048, SPS054, SPS061, SPS082, SPS106, SPS107, SPS113, SPS129, SPS136, SPS140, SPS151, SPS168, SPS172, SPS178, SPS184, SPS197, SPS198, SPS214, SPS215, SPS216, SPS219 \\
& Education & SPS038, SPS058, SPS101, SPS146, SPS187, SPS204 & SPS020, SPS072, SPS116, SPS120, SPS152, SPS206, SPS207, SPS218 & SPS089, SPS096, SPS115, SPS151, SPS163, SPS198, SPS199 \\
& Outreach & SPS002, SPS031, SPS037 & SPS078, SPS112 & SPS010, SPS063, SPS114 \\
\midrule
\multirow{10}{*}{\textbf{Q2}}
& Software development & SPS002, SPS037, SPS038, SPS044, SPS053, SPS058, SPS101 & SPS078, SPS120, SPS183, SPS195 & SPS008, SPS010, SPS028, SPS096, SPS172, SPS215 \\
& Computational thinking & SPS187 & SPS116 & SPS115, SPS198 \\
& Parallel computing & & SPS020, SPS134, SPS223 & \\
& Data analysis & SPS037, SPS071, SPS157 & SPS183, SPS209 & SPS005, SPS028, SPS045, SPS061, SPS082, SPS129 \\
& Artificial Intelligence & SPS073 & SPS209 & SPS082 \\
& Computer networks & SPS187 & SPS094 & SPS010, SPS019, SPS048, SPS106, SPS113, SPS198, SPS216, SPS219 \\
& IT infrastructure & SPS003, SPS007, SPS031, SPS037, SPS038, SPS039, SPS070, SPS073, SPS083, SPS092, SPS137, SPS143, SPS145, SPS146, SPS155, SPS176, SPS177, SPS187, SPS204 & SPS020, SPS027, SPS029, SPS055, SPS066, SPS067, SPS068, SPS078, SPS081, SPS094, SPS112, SPS117, SPS126, SPS134, SPS152, SPS167, SPS173, SPS174, SPS183, SPS205, SPS206, SPS207, SPS218, SPS223 & SPS019, SPS021, SPS030, SPS032, SPS033, SPS036, SPS048, SPS054, SPS082, SPS089, SPS096, SPS106, SPS107, SPS115, SPS129, SPS136, SPS140, SPS151, SPS163, SPS168, SPS172, SPS178, SPS184, SPS197, SPS198, SPS199, SPS214, SPS215, SPS216, SPS219 \\
& HPC & SPS083 & SPS027, SPS134 & SPS008, SPS114, SPS129, SPS178 \\
& Security & SPS064, SPS070, SPS083, SPS092, SPS155, SPS157, SPS192 & SPS081, SPS093, SPS094, SPS126, SPS153, SPS165, SPS183, SPS221, SPS226 & SPS082, SPS129, SPS214, SPS219 \\
& Cloud computing & SPS002, SPS003, SPS031, SPS070, SPS071, SPS080, SPS137, SPS143, SPS146, SPS177 & SPS029, SPS055, SPS126, SPS173 & SPS019, SPS030, SPS032, SPS033, SPS045, SPS136, SPS163, SPS197, SPS214, SPS216 \\
\bottomrule
\end{longtable}
\end{center}



\newcolumntype{Z}{>{\centering\arraybackslash}p{3cm}}

\begin{longtable}{Z Y Y Y}
\caption{Classification of SPS studies by technology of \textbf{VBC} and academic dimension}\label{tab:clasificacion-topicos} \\

\toprule
\textbf{Topics} & \textbf{Education} & \textbf{Research} & \textbf{Outreach} \\
\midrule
\endfirsthead

\toprule
\textbf{Topics} & \textbf{Education} & \textbf{Research} & \textbf{Outreach} \\
\midrule
\endhead

CRI-O & & SPS068, SPS083 & \\
\\
Containerd & & SPS066, SPS068, SPS083, SPS223 & \\
\\
Docker & SPS020, SPS038, SPS042, SPS058, SPS072, SPS089, SPS096, SPS101, SPS115, SPS116, SPS120, SPS124, SPS152, SPS187, SPS198, SPS199, SPS204, SPS206, SPS207, SPS218 
& SPS002, SPS004, SPS005, SPS007, SPS008, SPS011, SPS017, SPS021, SPS030, SPS039, SPS040, SPS041, SPS043, SPS044, SPS045, SPS046, SPS048, SPS049, SPS051, SPS053, SPS054, SPS055, SPS059, SPS060, SPS061, SPS065, SPS066, SPS071, SPS074, SPS079, SPS080, SPS081, SPS083, SPS093, SPS097, SPS099, SPS100, SPS102, SPS103, SPS104, SPS105, SPS106, SPS107, SPS119, SPS122, SPS124, SPS126, SPS129, SPS133, SPS153, SPS155, SPS172, SPS173, SPS174, SPS176, SPS177, SPS180, SPS182, SPS187, SPS188, SPS191, SPS192, SPS197, SPS198, SPS205, SPS209, SPS216, SPS219, SPS220, SPS221, SPS225, SPS226 
& SPS002, SPS037, SPS063, SPS078, SPS099, SPS112, SPS114, SPS220 \\
\\
Firecracker & & SPS107, SPS205 & \\
\\
Google gVisor & & SPS107, SPS184, SPS205 & \\
\\
Hyper-V containers & & SPS068 & \\
\\
Kata Containers & & SPS184, SPS205, SPS224 & \\
\\
LXC & & SPS066, SPS068, SPS083, SPS157 & \\
\\
LXD & & SPS068, SPS083 & \\
\\
OpenVZ & & SPS083 & \\
\\
Podman & & SPS007, SPS046, SPS060, SPS068, SPS083, SPS129, SPS174 & \\
\\
Rkt & & SPS068, SPS083 & \\
\\


Singularity & & SPS041, SPS060, SPS068 & \\
\\
Udocker & & SPS027, SPS068 & \\
\bottomrule
\end{longtable}

\begin{longtable}{Z Y Y Y}
\caption{Classification of SPS studies by IT domain and academic dimension}\label{tab:clasificacion-sps} \\

\toprule
\textbf{Topics} & \textbf{Education} & \textbf{Research} & \textbf{Outreach} \\
\midrule
\endfirsthead

\toprule
\textbf{Topics} & \textbf{Education} & \textbf{Research} & \textbf{Outreach} \\
\midrule
\endhead
\addlinespace
Data analysis &  & SPS001, SPS005, SPS028, SPS045, SPS061, SPS071, SPS082, SPS129, SPS157, SPS183, SPS209 & SPS037 \\
\addlinespace
Blockchain &  &  & SPS063 \\
\addlinespace
Cloud computing & SPS146, SPS163 & SPS002, SPS003, SPS012, SPS015, SPS018, SPS019, SPS025, SPS026, SPS029, SPS030, SPS032, SPS033, SPS043, SPS045, SPS055, SPS056, SPS069, SPS070, SPS071, SPS079, SPS080, SPS084, SPS085, SPS087, SPS091, SPS099, SPS109, SPS111, SPS126, SPS136, SPS137, SPS143, SPS146, SPS149, SPS173, SPS177, SPS179, SPS185, SPS193, SPS194, SPS197, SPS202, SPS210, SPS213, SPS214, SPS216, SPS217, SPS222 & SPS002, SPS031, SPS099, SPS213 \\
\addlinespace
Parallel computing & SPS020 & SPS017, SPS134, SPS223 & \\
\addlinespace
Software development & SPS038, SPS042, SPS058, SPS096, SPS101, SPS120 & SPS002, SPS008, SPS010, SPS015, SPS022, SPS028, SPS043, SPS044, SPS053, SPS086, SPS098, SPS100, SPS118, SPS133, SPS172, SPS183, SPS195, SPS215, SPS224 & SPS002, SPS010, SPS037, SPS078 \\
\addlinespace
HPC &  & SPS008, SPS014, SPS017, SPS018, SPS027, SPS041, SPS062, SPS083, SPS090, SPS098, SPS121, SPS129, SPS134, SPS178, SPS194, SPS200 & SPS114 \\
\addlinespace
Artificial intelligence & SPS072 & SPS011, SPS023, SPS027, SPS030, SPS040, SPS051, SPS053, SPS059, SPS073, SPS077, SPS080, SPS082, SPS095, SPS142, SPS148, SPS149, SPS154, SPS161, SPS169, SPS170, SPS177, SPS183, SPS209 & SPS037, SPS078, SPS181 \\
\addlinespace
Computational thinking & SPS042, SPS115, SPS116, SPS187, SPS198 & SPS187, SPS198 & \\
\addlinespace
Computer networks & SPS139, SPS187, SPS198 & SPS010, SPS019, SPS046, SPS048, SPS094, SPS103, SPS105, SPS106, SPS110, SPS113, SPS132, SPS159, SPS164, SPS187, SPS198, SPS216, SPS219 & SPS010 \\
\addlinespace
Security &  & SPS010, SPS019, SPS046, SPS048, SPS094, SPS103, SPS105, SPS106, SPS110, SPS113, SPS132, SPS159, SPS164, SPS187, SPS198, SPS216, SPS219 & \\
\addlinespace
IT infrastructure & SPS020, SPS038, SPS075, SPS089, SPS096, SPS115, SPS124, SPS146, SPS151, SPS152, SPS163, SPS187, SPS198, SPS199, SPS204, SPS206, SPS207, SPS218 & SPS003, SPS004, SPS007, SPS009, SPS011, SPS012, SPS014, SPS017, SPS018, SPS019, SPS021, SPS023, SPS024, SPS025, SPS026, SPS027, SPS029, SPS030, SPS032, SPS033, SPS034, SPS036, SPS039, SPS046, SPS047, SPS048, SPS049, SPS051, SPS052, SPS054, SPS055, SPS056, SPS057, SPS060, SPS062, SPS066, SPS067, SPS068, SPS069, SPS070, SPS073, SPS074, SPS075, SPS076, SPS077, SPS079, SPS081, SPS082, SPS083, SPS084, SPS085, SPS087, SPS088, SPS090, SPS091, SPS092, SPS094, SPS095, SPS099, SPS100, SPS102, SPS103, SPS104, SPS105, SPS106, SPS107, SPS109, SPS110, SPS111, SPS117, SPS119, SPS121, SPS122, SPS123, SPS124, SPS125, SPS126, SPS129, SPS130, SPS131, SPS132, SPS134, SPS135, SPS136, SPS137, SPS140, SPS143, SPS144, SPS145, SPS146, SPS148, SPS149, SPS150 & SPS031, SPS037, SPS078, SPS099, SPS112, SPS181, SPS208, SPS211, SPS213, SPS220 \\
 & & SPS151, SPS154, SPS155, SPS156, SPS159, SPS160, SPS164, SPS167, SPS168, SPS169, SPS170, SPS171, SPS172, SPS173, SPS174, SPS175, SPS176, SPS177, SPS178, SPS179, SPS180, SPS182, SPS183, SPS184, SPS185, SPS186, SPS187, SPS188, SPS189, SPS190, SPS196, SPS197, SPS198, SPS200, SPS201, SPS205, SPS208, SPS210, SPS212, SPS213, SPS214, SPS215, SPS216, SPS217, SPS219, SPS220, SPS222, SPS223, SPS224, SPS225 & \\
\addlinespace
\bottomrule
\end{longtable}

\twocolumn
\begin{figure}[htbp]
    \centering
    \includegraphics[width=0.5\textwidth]{resources/images/resultados/SPS-topics-QI.png}
    \caption{SPS by quality indices, topics, and research questions}\label{fig:SPS-topics-QI}
\end{figure}


\begin{center}
    \includegraphics[scale=0.3]{resources/images/resultados/Kw.png}
    \captionof{figure}{SPS keyword co-occurrence analysis}
    \label{fig:SPS-kw}
\end{center}

\setlength{\tabcolsep}{8pt} % separación horizontal entre columnas

%sub-subseccion 4.6.2
\subsubsection{Word Cloud Visualization}
Figure~\ref{fig:SPS-wordcloud} presents the keyword cloud generated from the 226~SPS (terms with frequency $> 1$). The three dominant clusters---\textit{Docker}, \textit{Container}, \textit{Kubernetes}, \textit{Cloud Computing} (61.6\%); \textit{Containerization}, \textit{Container Orchestration}, \textit{Virtualization}, \textit{Microservices} (13.08\%); and \textit{Performance evaluation}, \textit{Edge computing}, \textit{Machine learning}, \textit{Security} (9.09\%)---reflect the current thematic structure of CBV research. Notably, terms related to education, teaching, and outreach are absent from the high-frequency clusters, confirming the finding that academic applications of CBV remain an under-investigated research area.
\begin{center}
    \includegraphics[width=0.5\textwidth]{resources/images/resultados/wordcloud.png}
    \captionof{figure}{Keyword cloud of the 226 SPS}
    \label{fig:SPS-wordcloud}
\end{center}
